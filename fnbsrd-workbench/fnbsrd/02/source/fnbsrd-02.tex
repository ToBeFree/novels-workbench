\part{Numeri}

\chapter{Ribbons}

Eis und Schnee, Eiszapfen und Schneestürme. Tod und Verderben.

»Du hast es vorhergesehen«, murmelte Vängä. Sie starrte auf die Überreste eines Dorfes, in dem nicht nur der Mond untergegangen war. »Und nichts getan.«

Das Silberschwert bot den nötigen Halt, aber es fand kein Ziel. Nicht heute.

Der treue Leser mag vermuten, hier habe ein Walddorf namens Cygnis sein rachmotiviertes Ende gefunden, doch die Ruinen lagen weit im Westen, jenseits der Kartengrenze des vorherigen Abenteuers. Luyten, eine der größten Städte im Urwald der Elfenberge, war Geschichte. Vängä, eine knapp 1,80 Meter große Elfe in grün-brauner Jagdkleidung, wischte mangels Tränen eine blonde Strähne beiseite. Hier gab es nichts zu betrauern, nur hunderte verwesende Gründe für ohnmächtige Wut. Chaos, so weit das Auge reichte; zerstörtes Porzellan und Kleidungsfetzen. Kein Gold mehr.

Überhaupt fehlte alldem sein ursprüngliches Leben. Vängä hatte das Dorf auf Umwegen erreicht, einige alte Rechnungen beglichen, stets um die Zukunft des eigentlichen Ankunftsorts gebangt. Der Aberglaube hatte in den Sternen gestanden, anderswo hätte man Mathematik dazu gesagt. Letztendlich fragte niemand mehr nach Vorsicht.

Aus dem letzten Abschnitt des Heimwegs wurde der Beginn einer weniger ziellosen Reise. Wo Brieftauben fehlten, musste eiliger Schritt herhalten – nach Last Hope.

\begin{center}
∞∞∞
\end{center}

Last Hope, die letzte Hoffnung, war ein unabhängiges Dorf am Fuß des Nabenbergs. Das Zentrum der Welt mit drei verfeindeten Städten, jede benannt nach einer Farbe, thronte über den niederen Hoffnungsträgern. Absurd suffigiert und von gottgleichen Königen regiert bestimmte die Politik des Grüntals, Rottals und Blautals das Handeln der Dorfbewohner. Manche Dörfer schlossen sich den Talstädten an, manche schlugen sich allein durch die Weltstürme. Krieg tobte lange nicht mehr auf dem Berg, sondern zwischen den Stellvertretern am Boden. Technologisch fortgeschritten, gar mit Buchgeld und Arbeitsautomatisierung, ging die Zivilisation der Städte zugrunde. Weniger nobel und stolz weniger dekadent wurde in Last Hope vor allem mit den eigenen Händen der Primärsektor am Laufen gehalten. Selbst Öbfüs, Hotelwirt in Last Hope, fühlte sich eher als Handwerker denn als Dienstleister, wenn er das Geschirr spülte.

Last Hope war zudem die Heimat einer äußerst gemischten Abenteurergruppe mit inzwischen zwei Drachen, von denen einer sogar fliegen konnte und deshalb silberne Schuppen gebildet hatte. Willkommen in einem Universum, in dem Talent und Interessen sich mit der Zeit in der Kleidungsfarbe niedersetzen; willkommen in einer Gesellschaft, in der die Irisfarbe des Gegenübers seine magische Hauptschule verrät. Rot wie das Feuer, blau wie das Wasser, grün wie Pflanzen und gelb wie der Blitz; silber wie Wind und weiß wie das Licht – oder, und das ganz ohne subtraktive Farbmischung, pechschwarz wie der Tod. Wie Dunkelheit und Verderben, Umtriebe und Verschwörungen, mit lilanem Touch wie die Schwerkraft oder infrarot wie ein Weltstein.

Fnbsrd gähnte. Seine Iriden waren pechschwarz, seine Pupillen leuchteten infrarot. Ersteres passte zu seiner einfarbigen Kleidung, Letzteres war sehr ungewöhnlich und durch eine Sonnenbrille verdeckt. Er war einer der wenigen, die den Umgang mit Weltsteinen perfektioniert hatten. Weltsteine waren heiße schwarze Kugeln, etwa faustgroß und perfekt rund, in denen sich eine fremde Welt befand. Durch Portale mit Weltsteinrezeptoren war es den Bewohnern der hiesigen Welt möglich, in die fremden Welten abzutauchen und dort Jahre zu verbringen, während draußen kein Tag vorbeizog. Manche behaupteten gar, die ganze Realität befände sich ebenfalls in einem Weltstein, und eine Widerlegung war ohne Ausgangsportal nur dem möglich, der ihn erschaffen hatte.

Nicht ohne Grund gruselte diese Fähigkeit seine Mitmenschen. Fnbsrd war aus der Stadt Blautal verstoßen worden, hatte daraufhin den Weg ins echte Tal angetreten und war in Last Hope nach anfänglicher Skepsis nun sogar mit einer Hochzeit empfangen worden. Aqua, eine Wassermagierin, blond und üblicherweise im blauen Kleid, war der Dunkelheit mit ganzem Herzen verfallen.

Aus Ruhe und glücklicher Zweisamkeit sollte zumindest in diesem Winter nichts werden; das hatte jemand im Westen beschlossen.

koroljow, der silberne Drache des Dorfes, sah der heranrauschenden Elfe mit weniger rauschender Vorahnung entgegen. »Vängä.« Er drückte gemeinsam mit der roten Drachenfrau curie die Flügel des Westtors nach außen. »Was führt dich zu uns?«

»Der Tod führt mich zu euch«, keuchte sie. »Luyten ist verstummt.«

»Bitte tritt ein«, bat curie. Sie rief MAGNUS hinzu, einen nach Jahrhunderten in allen Schulen einigermaßen gebildeten Zwerg mit Vorliebe zur Windmagie.

Bänke am Dorfzentrum, durch curie vorsichtig vom Frost befreit, boten Platz für einen ersten Austausch. Im Hintergrund wurden Fnbsrd und Aqua informiert und traten langsam hinzu.

»Von Wölfen«, wiederholte koroljow ungläubig. »Wieso?«

»Nicht aus eigenem Antrieb«, so viel stand für Vängä fest. »Es gibt weit oben im Elfengebirge einen verrückten Einsiedler, der Wölfe kontrolliert.«

»Das klingt nicht so, als habe dies zum ersten Mal zu Problemen geführt.«

»Richtig«, bestätigte die Elfe sofort. »Man hat den Trottel damals in seine Schranken gewiesen, recht nachdrücklich und physisch. Wenn wieder Wölfe unterwegs sind, dann mindestens teilweise mit Vergeltungsabsicht.«

»Rache gegen Rache, Zahn um Zahn, eine Gewaltspirale, an der man sich nicht beteiligen muss«, murrte MAGNUS.

Vängä spuckte gehässig zur Seite. »Dreihundert Waldläufer sind gegen ihren Willen daran beteiligt \emph{worden}«, gab sie zu bedenken.

»Mhm«, machte der Zwerg.

Lange sprach niemand ein Wort, dann trat Aqua nach vorne. »Du glaubst nicht, dass das eine einmalige Aktion war?«

Fast war es, als leuchte das Silberschwert des Gastes auf. »Das ist wichtig?!«

»Nun«, murmelte Fnbsrd, »ich wage kaum zu fragen, was aus Cygnis geworden ist.« Dessen Bewohner hatten ihm und den anderen Anwesenden beim letzten Abenteuer aufgelauert; Vängä war anschließend »auf der Durchreise« bei den Verrätern vorbeigelaufen.

»Cygnis ist ein schönes Dorf. Luyten ist kein Dorf mehr. Luyten ist tot.«

»Kannst du uns das Ausmaß des Angriffs ein bisschen genauer beschreiben?«, bat curie.

»Das Gold fehlt, die Lebewesen sind nicht mehr buchstäblich, alles stinkt nach Abgrund. Viel mehr gibt es dazu nicht zu sagen.«

»Wie sehen die Dorfmauern aus?«

»Luyten hat keine Mauern. Luyten ist eine offene Waldsiedlung. War. Es hätte ohnehin jede Mauer gesprengt. Nun zieren nur noch Scherben den Boden.«

»Das klingt recht gefahrlos für uns«, fand MAGNUS. »Wenn die Wölfe hierher kommen, lassen wir sie vor die Steine rennen. Sie können sich am Tor die Zähne ausbeißen.«

»Zumindest Holztore«, widersprach Vängä, »haben die Wölfe nicht aufgehalten. Überhaupt ging das alles nicht ohne Zauber zu.«

»Meinst du, der Einsiedler – wie heißt er überhaupt? – war bei dem Überfall anwesend?«

»Nicht zwingend. Der Elf ist im Dunkelgrün sehr bewandert und kann Tiere zu Magiern machen. Oder mindestens zu magischen Werkzeugen.«

»Unschön.«

»Ja. Also, seid ihr dabei?«

»Wie bitte?«, fragten mehrere Stimmen gleichzeitig. curie ergänzte: »Was MAGNUS von einer Gewaltspirale erzählt hat, wirkt wie ein vernünftiger Einwand. Wir können uns verteidigen, wir können dich verteidigen, wir können die Lage beobachten. Wenn die Situation überhand nimmt, können wir vielleicht zur Wurzel ausziehen und sie …«

Vängä lachte bitter. »Ausreißen ist das Wort, werte curie«, spottete sie; »ausreißen, wie Nicht-Pazifisten es eher heute als morgen tun würden.«

»Du möchtest den Elfenbergen den Frieden bringen«, stellte die Drachin fest, und als sie schmunzelnd einen zustimmenden Blick geerntet hatte, »notfalls mit Gewalt.« Sie sah der Elfe tief in die silbernen Augen; ihre eigenen leuchteten rot.

Diesmal sprang Fnbsrd ihr zur Seite. »Die Gewalt, falls denn überhaupt welche notwendig ist« – ein skeptischer Blick von Aqua mit Zügen von Sarkasmus –, »richtet sich nicht gegen ihre Profiteure. Gegen keinen einzigen davon.«

»Ich weiß, dass du den Indikativ recht gerne gezielt einsetzt«, ließ Aqua seine Rhetorik platzen. »Du kannst dabei aber nur für dich selbst sprechen.«

»Und für mich«, gab koroljow zu. »Ich habe lange genug in Gefangenschaft gelebt und von meiner Befreiung profitiert, um sie anderen zu versagen. Wenn Unrecht geschieht, halte ich dagegen; Schuld an Eskalation hat das Böse stets allein.«

»Für mich ebenfalls«, meldete sich MAGNUS für die Mission. »Irgendjemand muss ja auf euch aufpassen.«

Vängä zog das Silbeschwert aus der gläsernen Hülle und hob es gen Himmel. Es leuchtete mit der Kraft der Umgebung und bot angemessenen Glanz. koroljow umschloss Vängäs Hand mit seiner linken Vorderpranke, MAGNUS hielt eine Faust unter das Gebilde. Dann stimmte auch Fnbsrd mit ein.

»Ihr dürft diesmal nicht damit rechnen, dass man euch mit Trompeten aus Jericho befreit«, stellte Aqua klar. »Ihr habt meine Sorgen und meine Kopfschmerzen, aber nicht meine Zustimmung.«

»Die brauchen wir bei aller Liebe auch nicht zwingend«, entgegnete Fnbsrd sanft. »Außerdem muss ja irgendjemand auf die Stadt aufpassen, während wir weg sind, nicht wahr?«

Nun war Aqua belustigt. »Das hast du letztes Mal auch gesagt, du Held.«

»War es denn falsch?« Fnbsrd lachte ebenfalls und streckte die Zunge heraus.

»Pff. Macht, dass ihr fortkommt!« Ein Kuss.

»Proviant?«

»Ist quasi schon bereitgestellt.« Zwei. Lächelndes Blinzeln. »Öbfüs kommt gleich vorbei. Eine Dose Pfeffer nur für dich.«

»Der Wirt weiß, was gut ist.«


\chapter{Frozen Winter Moonlight}

Aufbruch war einfach, der Anfang schwer. Weitestgehend vegetationslose Ebene war von einer dünnen Schneeschicht bedeckt und dominierte vorerst das Weltbild. »Was liegt im Westen?«, erkundigte sich Fnbsrd.

Diesmal war es MAGNUS, der eine Karte bot. »Luhman, ein kleines Doppeldorf. Da sind zwei Dörfer quasi zusammengewachsen, mit der Zeit.«

»Spannend. Und weiter?«

»Passenderweise ein Dorf namens Wolf.«

»Omen.«

»Nach Wolf erreichen wir Lalande, dann Majoris. Letzteres ist vergleichsweise groß, aber nicht so groß wie Alpha.«

»Und was ist das da?«

»Ross. Das umgehen wir mit recht großem Bogen, weil dort dem grünen König gehuldigt wird.«

»Apropos!« Hier hakte Vängä ein. »Es würde mich nicht wundern, falls der seine Finger im Spiel hat.«

»Unfug«, widersprach MAGNUS sofort. »Dann hätte ich andere Waffen mitgenommen und mich ausgiebiger verabschiedet.«

»Keks.«

»Auch ein Stadtname?«, wollte Fnbsrd wissen.

»Nee, MAGNUS ist ein Keks.« Vängä lachte. »Ein richtiger Weichkeks.«

»Vängä darf das«, erklärte MAGNUS. »Unterhalb einer gewissen Intelligenzlinie genießt mein Gefolge Narrenfreiheit.«

Bevor die Elfe sich darüber echauffieren konnte, nahm Fnbsrd mit ihm die Karte weiter unter die Lupe.

»Ran und Aquarii, letzteres an der Grenze zu den Elfenbergen. Wer Aquarii auf der Westseite verlässt, befindet sich bereits im großen Urwald.«

»Ich hoffe doch sehr, Aquarii steht noch«, bat Fnbsrd um Bestätigung.

»Oh, das ist nicht garantiert.« Vängä malte mit ihren Händen düstere Visionen auf Augenhöhe in die Ferne. »Es ist aber wahrscheinlich«, gab sie dann zu und nahm die Hände wieder herunter.

»Aquarii also, letzter Stopp vor der Ungewissheit, wäre ein guter Ort für eine Rast.«

In allgemeine Zustimmung gab koroljow vorsichtig räuspernd zu bedenken, Walddörfer seien häufig nicht für Drachen gebaut.

»Oh, du musst verzeihen«, entgegnete Vängä, »aber das hat nichts mit Abneigungen zu tun. Die Leute in den Bergen sind äußerst gastfreundlich. Mit steigender Höhe werden die Bäume zudem deutlich lichter.«

»Ich werde notfalls über den Wald hinweg fliegen und einen Landeplatz finden«, äußerte sich der Drache recht zuversichtlich.

Seine Begleiter waren ein wenig um seine Sicherheit besorgt; MAGNUS gab still zu verstehen, dass er in diesem Fall zum ersten Mal auf einem Drachen reiten und dessen Lager verteidigen würde. Fast musste er zugeben, er hoffe auf dieses Szenario.

\begin{center}
∞∞∞
\end{center}

»Tempel der Liebe?« Fnbsrd blickte skeptisch dem Dorfeingang von Majoris entgegen.

Vängä lachte. »Majoris ist eine sehr … weltoffene Stadt. Wir werden hier garantiert den einen oder anderen Mitstreiter finden.«

»Ein Haufen komischer bunter Vögel ist das«, murrte MAGNUS in gespielter Griesgrämigkeit. Vängä schlug ihm lachend mit der flachen Hand gegen den Rücken.

»Wenn du das sagst!«, lachte koroljow mit. Er breitete seine Flügel aus und ließ Luft über den Kiesweg rauschen: »Last Hope für euch. Vier Tickets für den Tempel.«

»Last Hope«, rief jemand von den Mauern nach innen. »Silberdrache und Befreier.«

Die Tore waren noch vor dem Ende der Ankündigung weit geöffnet. »Seid willkommen in unserem kleinen Fort«, bat eine Menschendame in Gelb. Sie trug einen Papagei auf der Schulter, der nicht weniger gelb war als ihre Kleidung. »Güdnäbend«, krächzte das Federtier am hellichten Mittag. »Güdnäbend!«

MAGNUS blickte grinsend zu dem Vogel empor. »Vielen Dank für den netten Empfang. Wie heißt er, wenn man fragen darf?«

»Das ist Doktor Jeep«, lachte die Frau. »Und ich bin Katja.«

Jeder stellte sich kurz vor. »Sagen Sie«, bat Fnbsrd sie anschließend, »haben Sie Lust, ein paar Blitze auf einen Tyrannen regnen zu lassen?«

»I wo«, wehrte sie freundlich, aber bestimmt ab. »Ich sorge hier für gutes Wetter und hübsche Beleuchtung. Bunter, als jede Lichtmagie es vermag. Das genügt mir.«

Fnbsrd nickte. »Leider werden wir das erst auf dem Rückweg genießen können. Wir sind ein wenig in Eile und müssen noch ein paar Dörfer weiter, bevor die Nacht anbricht.«

»Güdnäbend!«

\begin{center}
∞∞∞
\end{center}

Wo Blitzmagie nicht weiterhalf, bot sich eine Botanikerin an. Der Tempel der Liebe war ein Begegnungszentrum in vielerlei Hinsicht, und diese war recht nützlich.

»Ich wollte schon immer einmal durch den Wald ziehen«, schwärmte Maia. »Und wir sollten Köväläs mitnehmen, der ist Sanitäter.«

»Ein professioneller Heiler fehlt tatsächlich«, gab MAGNUS zu, was ihm trotz seiner eigenen Fähigkeiten nicht schwerfiel. »Wo finden wir ihn?«

Maia grinste. »Der schläft die letzte Nacht aus.«

»Ungut«, befand Vängä. »Es ist Mittag und die Zeit drängt. Wenn er mitkommen will, soll er nachkommen. Wir sehen uns dann heute Abend in Aquarii, der Stadt am Fuß der Elfenberge.«

»Ich gebe ihm Bescheid«, versicherte Maia. Sie verschwand kurz und kehrte mit einem verschlafenen Elfen zurück.

»Natürlich helfe ich den Waldläufern«, stimmte er sofort zu. »Wo muss ich unterschreiben?«

»Hier«, rief MAGNUS von unten. Er reichte dem Sanitäter einen Füllfederhalter, Tinte und einen Zettel mit der Aufschrift »Kreditvertrag für eine Waschmaschine«.

»Hä?«

»Spar dir den Unfug, zieh dir Waldkleidung an, pack Proviant ein, sei in fünfzehn Minuten wieder hier«, bat Vängä. Sie nahm den Zettel an sich. »Und lass dich von MAGNUS nicht zu sehr auf die Schippe nehmen.«

»Das klingt nach einem Plan«, fand Köväläs. »Ich heiße übrigens Köväläs.«

»Ich nicht«, entgegnete Vängä. »Vierzehneinhalb.«

\begin{center}
∞∞∞
\end{center}

Sechzehn Minuten und zwei Sekunden später wurde die obligatorische Vorstellungsrunde im Gehen nachgeholt. Die vorgebliche Eile war bewusst ein wenig übertrieben worden, aber Maia und Köväläs gestalteten ihren Abschied auch aus eigenem Interesse kurz und schmerzfrei. Im Tempel der Liebe konnte man Monate verbringen, ohne sich dessen vollständig bewusst zu sein. Der Weg durch die Außentür war wie das Erwachen aus einem Traum und klang nach Aufbruch.

»Aus dem Blautal?«, fragte Köväläs erstaunt.

»Das ist sehr lange her«, schränkte Fnbsrd die Magnuslegende sofort ein.

koroljow malte sich ein kleines Paradies im Wald aus. »Warst du schon einmal in Aquarii?«

»Noch nie«, entgegnete der Elf. »Es gab bisher keinen Grund dazu.«

»Das ist natürlich ein Argument«, gab der Drache zu.

\begin{center}
∞∞∞
\end{center}

Abendsonne über Aquarii.

Breite Holzpfähle vor Holzhäusern, und ein Tor.

»Wir lassen nach Einbruch der Dunkelheit niemanden mehr ein«, erklärte jemand von schräg oben.

Die Dunkelheit kippte leicht nach vorne und schielte der Stimme entgegen. »Wie schade. Draußen werden uns die Wölfe fressen.«

Wölfe waren ein schwieriges Stichwort am Elfenberg. Die meisten Bewohner waren alt genug, um sich an die damalige Geschichte zu erinnern, ohne einer Fortsetzung entgegenzustreben. Auf die Anspielung folgte daher vorerst gar keine Antwort.

»Großartig«, lobte MAGNUS seinen Kollegen. »Wir werden heute Nacht von Wölfen gefressen.«

»Unwahrscheinlich«, behauptete Vängä laut genug, um hinter den Kulissen gehört zu werden. »Wir haben schließlich kaum Gold dabei.« Eine unerhörte Provokation vor einem garantiert Gold lagernden Dorf. Der Eingang zum Elfengebirge war ein Touristenmagnet, Aquarii eine reiche Bastion.

Es vergingen vielleicht zwanzig Minuten in geringer Anspannung, bis das zu erwartende Resultat den Gästen einen Eingang bereitete. Das Tor war offen.

»Herzlich willkommen in Aquarii. Ihr werdet gebeten, ohne Umweg die Eukalyptushalle im Zentrum aufzusuchen.«

»Vielen Dank. Sehr gern.« MAGNUS ging erhobenen Hauptes voran; jeder Schritt verströmte bodengebundene Autorität.

Die Halle, ein Palast aus rotbraunem Fremdholz, war die Quelle der aquarischen Legislative, einer Zusammenkunft der Dorfvertreter, Demokratiezentrum des umgebenden Waldes.

»Legenden sterben nie. Sie werden ein Teil von dir.«

»\emph{Du} warst schon einmal hier.« Fnbsrd war überrascht und wenig verwundert zugleich.

»Es wird noch mehrere Winter dauern, bis wir gemeinsam Orte jenseits meiner Vergangenheit erreichen.« MAGNUS lächelte, Fnbsrd hob die Augenbrauen. »Und dann stehe uns die Freiheit bei. Die Unendlichkeit ist stellenweise grausam und tödlich.«

Nun bewegte sich die Sonnebrille ein Stück nach vorne. »Keine Sorge, ich bin dann ja bei dir.«

MAGNUS verschluckte sich und öffnete verstört den Halleneingang. Seinem Beispiel folgend betraten Fnbsrd, Maia, Köväläs und Vängä den Eingangsflur und blieben vor einem Empfangstresen stehen. Hinter ihnen blickte neugierig ein silberner Drachenkopf durch die Tür. »Notfalls kann ich der Anweisung ebenfalls mit ganzem Körper folgen«, schätzte koroljow. »Alternativ gehe ich davon aus, dass der gute Wille und der Kopf zählen.«

»Bitte entschuldigen Sie!« Ein Sekretär lief errötend aus einem Nebenraum herbei. »Wir haben selten, aber sehr gern Drachenbesuch. Sie sind in jeder Form willkommen in diesem Haus und diesem Dorf.«

koroljow lächelte und nickte. »Das passt schon. Machen Sie sich keinen Aufwand. Der Herr in Schwarz möge mich vertreten.«

Nun fühlte sich MAGNUS endgültig düpiert, ließ sich aber nichts davon anmerken. Er griff hastig nach der Initiative und stieß die Wahrheit wie einen Speer in den Raum. »Die Wölfe sind zurückgekehrt. Luyten ist gefallen.«

Dass Luyten dem Gastgeber ein Begriff war, lag nicht an der Reichweite der hiesigen Versammlung. Luyten war riesig ... gewesen? Was Luyten niedermähte, konnte Aquarii im Handstreich überrollen. »Luyten ist gefallen?«

»Alle Zeichen eines Raubüberfalls sind vorhanden«, berichtete Vängä. »Es mangelt an Gold und Überlebenden. Eine Meute hat sich satt gefressen, genau wie damals.«

»Ihr vermutet sicherlich, Atrokius stecke dahinter.«

»Der Modus Operandi spräche dafür, vieles bisher für unumstößlich Gehaltene jedoch dagegen.«

Der Sekretär besann sich seiner Dienereigenschaft und schüttelte den ersten Eindruck ab. »Bitte folgt mir. Es kann einen Moment dauern, bis wir die Versammlung zu so später Stunde einberufen haben.« Er öffnete die Haupttür des Innenraums; breite Holzflügel gaben den Blick auf einen runden Tisch frei. Mindestens hundert Lebewesen konnten hier Platz finden – notfalls auch Drachen. Während die Parlamentarier einberufen wurden, erhielt koroljow Zugang durch eine offenbar optionale, nicht tragende Wand.

\begin{center}
∞∞∞
\end{center}

Dass Handlungsbedarf bestand, war recht leicht zu vermitteln; dass dieser außerhalb des Dorfes lag, nicht. Ein Feuermagier ließ sich nach zweistündiger Verhandlung dazu herab, die Abenteuergruppe zu erweitern. Alle anderen planten mit absurd-naiver Ernsthaftigkeit die Verteidigung ihrer Holzwände gegen einen Wolftsunami.

»Wir könnten einen Schild aus Wind um das Dorf errichten«, schlug eine besonders clevere Elfe vor.

»Klar«, spottete Vängä. »Wo Holz nicht stabil genug ist, nehmen wir stattdessen einfach Luft.«

»Hast du einen besseren Vorschlag?«

»Ja: Ihr könntet euch mit zwanzig Zahnärzten an einer Wurzelziehung beteiligen.«

Die Diskussion lief im Kreis und führte zu keinem sinnvollen Ergebnis; irgendwann verließ die Lasthoper Delegation geschlossen den Eukalyptussaal. Die Bewohner von Aquarii konnten die Vorstellung nicht verarbeiten, während eines erwartbaren Angriffs auf ihr Dorf nicht anwesend zu sein.

»Das war ja fast erwartungsübertreffend produktiv«, murrte Fnbsrd, als die Tore hinter ihm zugefallen waren. Drinnen war man wohl noch immer schweigend ins Gespräch vertieft.

Als »Kömbüsö« stellte sich der Feuermagier vor. Er trug ein rotes T-Shirt und eine rote Schnittschutzhose, hatte leuchtend rote Augen und schien von innen zu glühen. Dazu im Kontrast standen Unsicherheit und manglende Praxiserfahrung – im Wald hatte es selten die Gelegenheit zum gefahrlosen Üben seiner Kunst gegeben.

Auch der Rest der nun sieben Personen starken Gruppe stellte sich kurz vor. »Wir sollten ohne Rast aufbrechen und jenseits des Dorfes lagern«, fand Vängä. »Die Diskussion hat mir bewusst gemacht, dass wir hier in größerer Gefahr als außerhalb sind.«

»Du glaubst wirklich nicht, dass unsere geballte Magierkompetenz eine Wolfshorde aufhalten könnte?«

»Hast du eine Vorstellung von Luytens Ex-Kompetenz?«

»Grob.«

»Die Wölfe haben keinen Funken Verstand davon übrig gelassen.«

Kömbüsö seufzte. »Gut. Wie gesagt, ich bin dabei und schließe mich euch an. Erlaubt mir aber bitte, zuvor eure Vorräte aufzufüllen.«

\begin{center}
∞∞∞
\end{center}

Mit gefüllten Vorräten und sanftem Bodenlicht zog die Hoffnung in die Nacht.

»Es ist ein stets wiederkehrendes Motiv«, stellte Fnbsrd fest, »dass der Einzelne zur Quelle des Übels auszieht und ihr in Untermacht die Welt entreißt.«

\iathought{In deinen Romanen zumindest}, flüsterte Aqua ihm in Gedanken zu. Fnbsrd zuckte zusammen, dann wurde er auch noch gedanklich geküsst.

»Alles okay bei dir?«, erkundigte sich Köväläs besorgt. Fnbsrd schien geistig abwesend zu sein und Selbstgespräche zu führen.

»Ich weiß nicht, wofür es sich zu kämpfen lohnt oder warum ich schreien muss«, stotterte der nur. »Denn ich bin nur eine Bruchstelle in einem Schloss aus Glas.«

Der Sanitäter legte ihm eine Hand auf die Stirn; eine Dampfwolke stieg auf und eine Salzkruste blieb zurück. »Okay…«

»Fnbsrd fragt sich bis heute, weshalb er sich nach Pavonis begeben hat«, murmelte MAGNUS.

Vängä nickte stumm; koroljow blickte sich interessiert um, die anderen wirkten ratlos. Der Zwerg löste das Rätsel vorerst nicht auf und genoss die Lage. Dann kehrte Fnbsrd zurück.

»Das wäre nicht nötig gewesen«, wandte er sich direkt an den Elfen aus dem Tempel. Das Salz fiel von allein in Blätterform zur Erde; wo es auftraf, wuchs für den Rest des Jahrhunderts nichts mehr.

Maia blickte auf den Boden und schüttelte sich. Durch Vängäs sanften Zauber leuchtete der Boden in sanftem Kristallweiß, bildete einen Pfad des Lichts durch die Finsternis. Die Botanikerin hingegen benötigte kein Licht, um Dürre wahrzunehmen.

»Wir sind bisher keinem spezialisierten Lichtmagier begegnet«, fiel Fnbsrd auf.

»Meinst du, ihr würdet euch verstehen?«, erkundigte sich koroljow lächelnd.

»Sehr!«

Der Drache legte den Kopf schief und blickte ihm zweifelnd gegen die Sonnenbrille.

»Also, zumindest einseitig.«

koroljow runzelte die Stirn. »Mich würde mal interessieren, wie viel Licht du verdrücken kannst, bevor du satt bist.«

»Mehr, als ein einzelner Magier jemals bieten könnte.« Fnbsrd tippte dem Drachen sanft mit einem Zeigefinger auf die Nüstern. »Aber ich glaube, deine Frage impliziert ein Missverständnis!«

Ein Luftzug verwirbelte Fnbsrds Haar; Lachen erklang ohne Mundbewegung. »Ich bin ganz Ohr.«

»Je mehr Licht ich erhalte, desto mehr Platz ist für den Rest des Lebens!«

»Das ist hohe Physik«, gab der Drache zu. Er blickte MAGNUS an, den er inzwischen vermutlich für allwissend hielt, und bat das Lexikon um eine Plausibilitätsprüfung. Es nickte nur.

»Und ist das gefährlich?«, wollte Maia wissen.

»Nein«, beschwichtigte der schwarze Magier sie sofort. »Das geschieht alles außerhalb unserer Welt.«

»Hör mal«, löste MAGNUS nun doch noch – räuspernd – seine Stille, »du hast vor einiger Zeit damit begonnen, Gravitationszauber zu erlernen.«

Fnbsrd nickte, verwundert über die Frage, aber eindeutig offen für Antworten.

»Hast du das getan, weil du in deiner alten Schule alle Zauber erlernt hast?«

Nun lachte der Mann in Schwarz. Das konnte nicht ausgerechnet MAGNUS’ Ernst sein. »Hör mal«, gab er also zurück: »Es gibt kein festes Spruchbuch, keine Magieliste, keinen Zauberkatalog. Das weißt du.«

MAGNUS nickte nur. Da musste deutlich mehr Text folgen.

»Mit Dunkelheit allein lässt sich kein Brot verdienen, kein Gegner überwältigen, kein Wolf stoppen. Kein… Drache befreien.« Er blinzelte koroljow zu, der ihm noch immer an den Lippen hing. »Also habe ich zunächst Heilzauber erlernt. Die hängen bekanntermaßen mit schwarzer Magie zusammen.«

»Ellen sagte mir, als Heiler seiest du nur äußerst eingschränkt talentiert. Bei Schwerkraft sieht das wohl anders aus?«

»Schwerkraft ist großartig«, stimmte Fnbsrd energisch zu. »Man kann sie mit den anderen Zaubern verbinden und Welten daraus formen.«

»Weltsteine.«

»Welten \emph{in} Steinen.«

MAGNUS schüttelte verwirrt den Kopf. »Worauf ich hinaus wollte: Wenn es keine feste Zauberliste gibt, kannst du Dinge erforschen, die in keinem Buch stehen. Du kannst quasi deine Lieblingsschule erweitern.«

Fnbsrd starrte den Zwerg eine ganze Weile an, bevor er durch Weitergehen einen Themenwechsel anstieß.

Vängä blickte sich um. »Wir sind inzwischen mindestens zehn Kilometer von Aquarii entfernt. Wenn ihr mögt, könnten wir dort drüben unser Lager aufschlagen.«


\chapter{It's 3 AM}

Die Nacht verlief ereignislos zwischen Tannen und Fichten. Als die ersten Sonnenstrahlen durch das Nadeldach brachen, erwachte der silberne Drache und nieste einmal ausgiebig dem nächsten Stern entgegen. Dann waren alle wach.

»Guten Morgen«, lachte Maia verschlafen. Sie hatte dichte Decken aus Efeuschichten geformt, die nun von ganz allein beiseite wichen. Um sie herum erwachte die Natur zu neuem Leben.

»Das nächste Dorf auf unserem Weg heißt…« MAGNUS kratzte sich am Kopf. »Haben wir überhaupt einen Plan? Vängä?«

»Es ist nicht ganz klar, ob überhaupt noch ein Dorf in den Bergen steht. Wenn sich die Welle des Bösen von oben herab ausbreitet, ist dort schon jede Hoffnung verloren.«

»Du möchtest zum Berggipfel?«

»Ich glaube, das wäre das klareste auf jeden Fall noch existierende Ziel.«

Mehrere nickten. Der Berggipfel, derzeit von Urwald verdeckt, erahnbar durch den Gradienten der Bodenhöhe, rief nach einem Besuch.

Ein kurzes Frühstück, vielleicht das letzte für lange Zeit, rundete den ansonsten recht unvorbereiteten Aufstieg ab. Karten hatten jenseits der Beständigkeit ihre Bedeutung verloren… fast.

»Mir hat einmal jemand erklärt, die Suche nach der Bergspitze endet je nach Suchmethode auf einem kleinen Hügel abseits des eigentlichen Bergs.« Mit diesen Worten zog MAGNUS nun doch die Karte hervor. Er zeigte auf kleinere Erhebungen zu beiden Seiten des Weges. »Das gilt aber nur, wenn man die höheren Punkte nicht aus der Ferne sieht.«

»Ich glaube, das wird uns nicht passieren«, widersprach Vängä. »Der Berggipfel ist dick schneebedeckt. Den erkennt man am Eis in seinen Schuhen. Wenn der Schnee fehlt und Bäume die Sicht versperren, sind wir nicht auf dem richtigen Berg.«

Das war eine gute Orientierungshilfe, fand koroljow. Der Wald wurde vorerst dichter und ihm langsam zu eng; er bahnte sich mit Vängäs Hilfe einen Weg in die Luft. Als er ein paar Runden über die Landschaft gedreht hatte, kehrte er krachend auf einen dadurch entstehenden Landeplatz zurück. Die anderen eilten zur neuen Lichtung. »Ich glaube, ich habe den Gipfel im Nebel erkannt.«

»Dann würde ich vorschlagen, wir fliegen dorthin«, stieß Vängä impulsiv heraus.

»Ich habe noch nie jemanden geflogen«, gab koroljow zu. »Mit einer Windmagierin könnte es funktionieren. Du kannst zuallermindest dein Eigengewicht kompensieren, nehme ich an.«

Das brachte ihm und ihr einige belustigte Blicke ein. Ja, das war eine Option. MAGNUS hatte sich insgeheim den Drachenritt erhofft, gab aber unumwunden zu, dass es wohl keinen besseren Passagier als die windbegabte Elfe gab.

»Ich bin auch noch nie geflogen worden«, stellte Vängä kurz klar. »Verlasse dich bitte nicht auf meine Erfahrung.«

»In Ordnung.« Der ganze silberne Körper zitterte nervös, eine halbe Minute lang regte niemand eine Kralle. Dann schoss das Gespann in den Himmel, als habe es nie etwas anderes getan.

Der arme Meisterzwerg wurde von dem Windsog fast von den Füßen gerissen. »Sapperlot Herrgottnocheins Himmel nochmal.« Er hielt sich würdewahrend einhändig an einem Baumstamm fest. Auch Köväläs hatte nicht mit dem Abflug gerechnet und kämpfte mit seiner Frisur.

»Steht dir«, log Maia lachend. »Kannst du ruhig so lassen.« Alle lachten.

»Das ist der Wahnsinn«, rief Vängä von oben. Sie surfte irgendwo zwischen zwei Wolken und gelösten Baumnadeln auf einem teilweise nur gedanklich an sie gebundenen Silberpanzer. Die verdrängten Wassertropfen übten stillen Protest in kalter Morgendusche. »Aieeeee!«

koroljow erwies sich als leidenschaftlicher Chaospilot, dem Stürme ungefähr so viel anhatten wie die Sonne den nach ihr benannten Blumen. Es zog ihn geradezu ins Gewitter hinein; seine Begleitung drückte mehrfach viel zu knapp die fliegenden Blitzpotenziale beiseite.

»Das ist jetzt aber schon ein wenig übertrieben«, bewertete sie die Achterbahnfahrt. Es rummste gewaltig zwei Meter schräg unter ihr. »Der Himmel war doch gerade noch blau.«

»Wir sind gleich am Ziel«, glaubte der Drache zu wissen. Er stürzte sich erneut in die Spannung, dann durchzog sie ihn zum ersten Mal wirklich. Der Flug geriet ins Wanken, die Flügel gehorchten nicht mehr ganz den Ideen des Kommandanten. Meuterei war das, auf hoher Wolke.

»Wir stürzen ab«, widerprach Vängä schreiend.

»Mag… sein…« Es ließ sich schlecht leugnen. »Wo sind wir überhaupt?«

Mit diesen Worten verabschiedeten die Wolken sich nach oben. Schnee, tatsächlich, Schnee und fehlende Vegetation dominierten die Landschaft… das Ding, das sich Landschaft nannte und doch weit und breit nur nach Gegend aussah. Wie auch immer: Kalt.

Vielleicht war es eine Eisschicht unter dem Schnee, vielleicht panisches Laufen im letzten Flugabschnitt. Ein Schlittern auf Stein und Wasser. Vängä sprang vom Luftschiff. Es war auf Grund gelaufen, keine Frage.

koroljow schüttelte den Schnee ab wie ein nasser Pudel. Inmitten weißer Eiskristalle kamen die Silberschuppen besonders schön zur Geltung; die Morgensonne glitzerte den Gestrandeten aufmunternd zu: Nicht alle Wüsten waren Sandwüsten.

\begin{center}
∞∞∞
\end{center}

»Warum genau haben wir die beiden jetzt nochmal allein zum Berggipfel fliegen lassen?«, wagte Fnbsrd eine vorsichtige Nachfrage. Der Sturm hatte sich gelegt, klare Gedanken tasteten sich einen Weg durch geistigen Nebel.

Lange fand niemand eine Antwort. »Weil sie das so wollten«, erkannte MAGNUS mäßig begeistert. »Sie werden das Rätsel nun allein lösen, die Welt in unserer Abwesenheit retten und mit froher Kunde zurückkehren. The End.«


\chapter{Ich kann nie wieder fliegen}

»Sind deine Pranken in Ordnung?«, erkundigte sich Vängä besorgt. Das Sonnenlicht biss in all seiner Schönheit stark ultraviolett an der Haut; sie war es, die sich in Gefahr befand.

Darauf wies koroljow dann auch direkt hin. »Ich bin in Ordnung. Für dich müssten wir Sonnencreme erfinden, oder du suchst dir Schutz vor der Strahlung.«

Die Elfe verkroch sich unter einem der Drachenflügel und murrte mit Blick auf blendenden Schnee herum. »Hier sind jedenfalls weder Einsiedler noch Wölfe.«

»Es geht noch viel weiter nach oben.«

»Kannst du mir versprechen, weniger verrückt zu fliegen? Dann könnten wir es noch einmal versuchen.«

koroljow lächelte. »Ich werde mich bemühen.«

Und dabei blieb es dann auch, denn seine Flügel gehorchten ihm nicht mehr.

\begin{center}
∞∞∞
\end{center}

»Dann wirst du erfrieren«, stellte Vängä recht nüchtern fest.

»Sehr langfristig, ja. Wenn ich mich zusammenkauere…« Der stolze Gigant rollte sich zusammen, der Begriff »Lindwurm« drängte sich auf. »…dann habe ich viel Volumen für wenig Oberfläche. So halte ich eine Weile durch.«

»Glaubst du.«

»Himmel, was soll ich sonst tun?«, rief koroljow ihr verzweifelt entgegen.

Darauf hatte Vängä keine Antwort.

\cleardoubleevenpage

\includepdf[pages=-]{z-include-XXXXXX.pdf}

\addtocontents{toc}{\protect\newpage}
% Neue Seite im Inhaltsverzeichnis

\part{Deuteronomium}

\chapter{In alle Ewigkeit}

»Eigentlich mag ich den Regen«, bekundete Fnbsrd. Die Bäume waren dicht und ließen wenig davon hindurch.

»In der Dosis finde ich Regen ebenfalls angenehm«, pflichtete Maia bei. »Und die Pflanzen lieben ihn natürlich.«

Kömbüsö rieb sich die Oberarme. Wo Regentropfen auftrafen, verdampften sie schnell. »Es wird euch wenig überraschen, dass ich Regen nicht allzu offen gegenüberstehe«, witzelte er.

»Das kann ich gut nachvollziehen«. Köväläs lächelte warm. »Jedes Unwetter geht vorbei.«

So leicht ließ sich MAGNUS nicht beruhigen. »Wie sieht es mit Schnee aus?«, wunderte er sich. Der Feuermagier musste doch gewusst haben, worauf er sich einließ.

»Ich glaube, mit Schnee kann ich recht gut umgehen. Die Feuermagier bei uns sind verantwortlich dafür, dass die Dorfstraßen schneefrei bleiben.«

»Na immerhin.«

\begin{center}
∞∞∞
\end{center}

»Bleibt stehen«, zischte Maia. Die Botanik hatte einen Knacks.

Etwa zweitausend Meter entfernt zerbrach Gras unter fliehenden Füßen. Die Bäume schluckten den Schall; die Bäume schluckten die Angst. Moos und Rinde fraßen jeden Hilferuf.

»Da vorne wird Geschichte zerrissen«, flüsterte die grüne Magierin.

»Geschrieben?«

»Zerrissen, Fnbsrd, zerstört.«

MAGNUS dröhnte durch das Unterholz. Er flüsterte womöglich aus Prinzip nicht. »Vängä und koroljow?«

»Niemals. Dafür ist der Blumentod zu verstreut. Zu viele Lebewesen fliehen gleichzeitig vor zu vielen anderen Lebewesen. Ich glaube, sie bewegen sich von uns weg.«

»Dann sollten wir hinterher laufen«, fand Köväläs.

»Meinst du, wir sind gemeinsam stärker als die Fliehenden?«

»Stärker als die Verfolger?«, fügte Fnbsrd hinzu. »Ohne Elfe und ohne Drachen.«

Der Zwerg hielt sich allein für stärker als jeden erdenklichen Bösewicht. »Das werden die Wölfe sein, auf frischer Tat. Vielleicht eilen sie zum Berggipfel, wo Vängä gerade ihr mit ihrem Silberschwert herumfuchtelt.«

Maia fand, das würde sich anders anfühlen. Sicher benennen konnte sie es nicht.

»Drei Stimmen für eine Verfolgung«, murrte Kömbüsö dann. »Es ist die einzige Spur, die wir haben.«

»Vielleicht kommt koroljow zurück und wir sollten am Abflugpunkt warten«, warf Maia ein.

»Mir wäre lieber, wir lassen den Wald hinter uns und die beiden sehen uns aus der Luft.«

Nicken aus mehreren Richtungen.

»Gut, dann sei es so.« Fnbsrd schritt voran und wurde sogleich von MAGNUS überholt, der ihn gönnerhaft bei der Hand nahm – und sich nach etwa zehn Sekunden daran verbrannte. Während er ohne Verlangsamung seiner Schritte zerknirscht die Blasen heilte, ersetzte Fnbsrd die abgelehnte Hand durch seine eigene Linke. Ein Weltstein entstand, ein Stock lag im Weg, ein Weltstein fiel. Und Maia schrie.

Die glühende schwarze Kugel brannte sich durch totes Laub, erhitzte das einige Zentimeter über ihr zum Stehen kommende T-Shirt, das gierig die Infrarotstrahlung aufsog. Kein Wolfram auf der Welt verarbeitete die Hitze so gut wie Fnbsrds Schwarz. Fast war der Schwarzmagier versucht, seine Kreation zu küssen oder die Sonnenbrille abzunehmen. Nur für ein paar Sekunden. Verlockend. Dann aber stieß ihn jemand rabiat zur Seite.

»Kömbüsö.« Fnbsrd stellte es eher unbewusst und ernüchtert fest, war in Gedanken bereits zehn Ebenen tiefer getaucht.

Kömbüsö. Der wusste immerhin mit Feuer umzugehen.

Kömbüsö für den Wald, Feuer mit Feuer, Vorbild für die schwedische Luftwaffe.

Der Feuermagier wies das Infrarot in seine Schranken, ohne es zu löschen. Fast absurd wirkte das Wachsen des Weltsteins.

»Jesus, was tust du?«, stotterte Fnbsrd. Der Wald schien vorerst gelöscht zu sein; MAGNUS hatte ein wenig Wasser beigesteuert und Maia erholte sich von ihrem Schock.

»Ich tue Umweltschutz. Du kannst den Wald nicht abbrennen, ohne CO₂-Zertifikate zu kaufen.«

Der Weltstein wuchs derweil auf das Doppelte des geplanten Volumens an und war selbst für Fnbsrds Geschmack ein wenig zu warm.

»Natürlich.« Fnbsrd richtete sich freihändig auf und war dankbar dafür, dass man ihm dabei nicht zu helfen versuchte. »Die Welt in meinen Händen freut sich jedenfalls sichtbar über dein Zuvorkommen.«

»Du hast sie auf Kosten der Umgebung künstlich klein gehalten«, stellte der Feuerelf fest.

Fnbsrd verzog den Mund. »Vielleicht habe ich einfach meine Hände geschützt.«

MAGNUS sah nicht ohne Genugtuung, dass der große Weltstein ihn überforderte.

»Wir haben keine Zeit zu verlieren«, fanden Fnbsrds Hände. Ihr Träger wandte sich an die verstörte Botanikerin. »Spürst du die Wölfe noch?«

Maia starrte ihn mit stumm geöffnetem Mund an, eher pikiert als entsetzt, eher protestierend als panisch.

»Bitte. Nicht für mich.«

Unter Schütteln verschwand der lokale Eindruck und der Sinn wurde wieder frei für die Ferne. »Sie bewegen sich kaum noch.«

Mindestens sechs Augen blickten sich gegenseitig an; mindestens sechs Beine liefen gleichzeitig los.

Durch dichtes Unterholz und bremsendes Gebüsch, inzwischen nicht mehr auf Tarnung und Stille bedacht, eilte ein heterogenes Rettungskommando den Fremden zur Hilfe. Rotschwarze Leidenschaft geleitet von grünem Überblick, im Hintergrund ein weißer Hoffnungsschimmer und ein undurchsichtiger Strippenzieher. Ein Keil stieß durch den Wald.

Maia spürte mit zunehmender Nähe die Einzelheiten auf dem fernen Boden. Fallende Körper ließen die Pflanzenwelt ächzen und die Fühlende weinen. »Wir kommen zu spät, um alle zu retten.«

Köväläs ballte die Fäuste zusammen. »Wie weit noch?«

»Ein Kilometer vielleicht«, schätzte sie.

Das waren vielleicht tausend Meter zu viel, fand der Heiler. Er trieb den Keil vor sich her, wagte jedoch mangels Angriffszaubern nicht, zur Spitze vorzudringen.

\begin{center}
∞∞∞
\end{center}

Zweihundert Wölfe, bereits ohne Magie viel zu viele auf einmal, besessen von dunkelgrüner Magie, manipuliert und benutzt, Werkzeuge einer höheren Macht. Zweihundert mal zweiundvierzig Zähne. Eine Kralle für jede Sekunde einer Stunde.

Sie standen den geflohenen Einwohnern eines Walddorfes gegenüber – zumindest war das bis zur Umzingelung noch ein eindeutiger Begriff gewesen. Der Wolfskreis beließ es auch nicht bei Drohungen und Maia irrte nicht.

Das Knacken im Unterholz war schneller als die Witterung. Zwei Elfen fielen Wolfsangriffen zum Opfer, der Rest fuhr herum.

»Es ist schlimmer als erwartet«, klang es aus dem Wald. »Es sind zu viele.«

Ein Teil der Wölfe, vielleicht hundert, vielleicht zweihundert, löste sich aus dem Außenkreis und drohte einen neuen zu bilden. Niemand wagte den ersten Angriff. Der Keil dehnte sich aus, wurde zu einer Linie, die den Halbkreis blockieren sollte. Ließ Maia und Köväläs zurück im Zentrum. Half nicht.

»Geteiltes Leid«, spottete Fnbsrd, »ist fast ein Kreis.«

»Das ist die richtige Einstellung«, murrte MAGNUS. Vor ihm fletschte ein weißer Wolf sein Fleischgebiss, blitzte mit Eckzähnen unter grün funkelndem Todesblick im Wissen seiner Rückendeckung. Der Zwerg riss sich erheblich zusammen, um eine Eskalation zu verhindern.


\chapter{Unter Schnee und Eis} % Musiktitel fehlt noch



\chapter{In the Air Tonight}

Die Erkenntnis, ebenbürtigen Gegnern gegenüberzustehen, erhielt mit fortschreitendem Nahgeplänkel eine gallige Beimischung aus der Ferne.

»Das war kein Krieger.«

Kömbüsös Hände waren von Flammenzungen umgeben, die nach Wolfsfell lechzten. Maias Stimme drang ziellos aus dem Kreis.

»Das war ein Elfenkind.«

MAGNUS antwortete, ohne den Blick von den Eckzähnen zu nehmen. »Dort vorne stirbt ein Dorf, das gnadenlos inklusive Fliehenden untergegangen wird. Wie in Luyten, so auf diesem Waldboden. Sinn- und maßlos. Es geht längst nicht mehr um Gold. Die Wölfe sind außer Kontrolle.«

Auch Fnbsrd fand neben seiner Weltkugel Zeit für eine kurze Antwort. »Du meinst, der Einsiedler steckt nicht mehr dahinter?« Das unsichtbare heiße Leuchten war den Wölfen nicht geheuer. Es half den umzingelten Dorfbewohnern wenig und wirkte nur vorübergehend gegen die Massen, die sich inzwischen mit den Neuankömmlingen befassten.

»Ich glaube, er hat die Kontrolle über seine Schöpfung verloren. Man sieht hier eindrücklich, warum magische Genmanipulation aus den Schulen verbannt wurde. Das tiefe Dunkelgrün ist keine gute Farbe für Geister, die man ruft.«

Als die Wölfe begannen, sich mit einer Schulklasse zu befassen, brannte in Fnbsrd eine Synapse durch. Violettes Schwarz umgab den Weltstein zwischen seinen Händen. Aus Funken wurden Blitze, die den Himmel verdunkelten. Der Stein fiel, der Boden fing Feuer, doch die Blicke aller Lebewesen gingen nach oben. Die Tageszeit war nicht erkennbar, der Himmel verschwunden. Eine Spirale aus Sternenstaub durchzog das Schwarz, das über jeder Nacht und jedem Tag hing. Durchsichtige Trennwände schoben sich quer von oben durch den Planetenboden, teilten Wald und Welt. Stellenweise reflektierte das Sternenglas das ermattende Licht. Scharfe Kanten blitzten jeweils kurzzeitig auf, Sturm und Regen peitschten durch das Chaos.

Zu Boden gestoßen, zur Seite geschleudert, der Schwerkraft hilflos ergeben verlor ein Jeder seinen Halt auf dem diesseitigen Waldboden. Kein Baum verhinderte den Fall zur Seite, kein Moos bremste den Querschlag.

Fnbsrd hob unkontrolliert die Welt aus ihren Achsen und ließ sie neben der Angel mit Schwung fallen. Es rummste gewaltig im Äther.

Als die Wolken des Verstandes sich lichteten, als der Nebel des Waldes abzog, als die Geräusche ihre Betäubung verloren und Maia wieder Grün unter den Füßen spürte, gab Gaia dem Schwarzmagier nur den Blick auf Hochhäuser frei.

Eine fremde Ampel regierte über den vorausliegenden Straßenbereich. Statt Rot und Grün stand dort nur eine Zahl in einem fremden Zahlensystem: Überlebenswahrscheinlichkeit beim Überqueren zum aktuellen Zeitpunkt.

In Promille. Ein Sportwagen schoss bremsenlos vorbei. Bahnrad für Motorsportler, Bremsen mit dem Rückwärtsgang. Selten.

Fnbsrd sprang ein Stück nach hinten, nur um beinahe die zweite Fahrbahn zu kreuzen. Ein Trippeln nach vorne: Willkommen auf deiner eigenen Verkehrsinsel.

\begin{center}
∞∞∞
\end{center}

\noindent \iaquote{»Jede hinreichend fortschrittliche Technologie ist von Magie nicht zu unterscheiden.«}\\
\noindent~– Arthur C. Clarke

\begin{center}
∞∞∞
\end{center}

\iathought{Man hätte es sich denken können}, sagte der Gast in Schwarz zu sich selbst. »Der Autor kann das besser als Fantasy.«

»Glaubt er zumindest«, piepste eine Maus. Piepste… Fnbsrd.

»Himmel!« Eine Maus. Natürlich. Vorsichtig bot Aqua ihrem kleinen Begleiter eine Hand auf den Boden an. Fnbsrd kletterte dankbar den Arm empor auf ihre Schulter. »Du brauchst noch einen Namen.«

»Den Namen gibt es noch nicht«, erinnerte ihn die Dame aus Last Hope. »Der Name wird nachträglich eingesetzt, wenn wir ihn gefunden haben.«

»›Aqua‹ wird es jedenfalls nicht werden«, stellte die Maus klar. »Dazu fehlt dir aktuell das Format.«

»Gut, dann setzen wir ab hier Platzhalter ein. Mein neuer Name ist SoulOfTheInternet.«

»Geht das auch kürzer?«

»Du kannst mich Soti nennen. Der Erzähler wird die Abkürzung in Großbuchstaben schreiben.«

Damit war Fnbsrd einigermaßen zufrieden. »Die Seele des Internets also.«

Und die Seele nickte glücklich.

\begin{center}
∞∞∞
\end{center}

»Wie kommen wir jetzt hinüber?«, wollte Fnbsrd wissen.

»Mit Glück \emph{und} Verstand«, hoffte der schwarz gekleidete Junge. Er blickte durch seine Sonnenbrille auf die vermutlich steigende Überlebenswahrscheinlichkeit, ohne ihr weiter über den Weg zu trauen als einem gezinkten Kartenstapel.

»Es scheint sich um ein Siebenersystem zu handeln. Beziehungsweise, es handelt sich um eines, sofern wir bereits alle Ziffern gesehen haben.«

»Das könnte auch bedeuten, dass die Wahrscheinlichkeit nie einen bestimmten Wert in einem viel höheren System überschreitet«, fachsimpelte SoulOfTheInternet ohne großes Nachdenken.

Fnbsrd war einigermaßen beeindruckt, behielt aber das letzte Wort: »Dann wäre die Zahl nicht zweistellig mit variierender erster Ziffer, ohne dass wir weitere Ziffern an zweiter Stelle gesehen hätten.«

Zwei Lasttransporter zischten vorbei. Ein Linienbus erinnerte den verängstigten Inselbewohner an den Begriff »Bus-Faktor«, als er einen Luftkubus vor sich her presste und einen Sog in Straßenrichtung erzeugte.

»Abstand von der Bahnsteigkante« forderte eine imaginäre Stimme ein.

»Woher nehmen, wenn nicht von der anderen Seite stehlen«, beklagte SOTI sich mit Blick nach oben. Er vergewisserte sich, dass die Maus sicher in der Kleidung festgekrallt war und hielt sie zusätzlich fest. Mit der Hand auf der Maus und ungeachtet der Ampel sprang er zu einem einigermaßen günstigen Zeitpunkt nach drüben.

»Ich glaube, das Geheimnis dieser Ampel heißt ›selbsterfüllende Prophezeiung‹ und ›Nocebo-Effekt‹.«

»Na und?«, fragte SOTI verwundert. »Es hat doch funktioniert.«

\begin{center}
∞∞∞
\end{center}

»Dem Leser fehlt die Einführung in diesen Komplex«, murmelte Aqua.

Die Maus hatte jedes Wort gehört und piepste ihr ins Ohr. »SoulOfTheInternet, wir sind hier, um das Mechanische Reich zu stürzen, bevor es zu groß wird.«

»Das mechanische Reich?«

»Ja, das ist der Witz. Mechanisch wird es erst durch unsere Manipulationen.« Fnbsrd kicherte.

»Wie?«

»Wir treten eine KI-Lawine los. Die Bewohner dieser Stadt sind faul; sie werden eines Tages durch ihre eigenen Diener ersetzt. Man muss kaum nachhelfen. Übrig werden Roboter bleiben, gegen welche die Protagonisten eines anderen Romans moralisch viel legitimer Krieg führen können als gegen unsere aktuelle Umgebung.«

Die Seele des Internets schnappte nach Luft. »Und wie rechtfertigen wir \emph{das}?«

»Durch die genetischen Experimente und Versklavungsaktionen dieses noch-nicht-mechanischen Reichs. Auch das ist allerdings ironisch.«

»So?«

Der kleine Nager nickte. »Ja. Denn diejenigen, die den Protagonisten eine Heimat bieten und mit ihnen gemeinsam die Galaxis retten, entstanden aus diesen Experimenten.« Unter seinen Pfoten schüttelte sich eine Schulter.

»Nochmal kurz zum Mitschreiben. Wir befinden uns in einer Großstadt mitten unter höchst unmoralisch agierenden Lebewesen, deren genetische und mechanische Kreationen ihnen durch unsere Mithilfe über den Kopf wachsen und die Großstadt einäschern werden.«

»Das ist…« Fnbsrd ging in Gedanken die Zukunft durch und suchte vergeblich nach einer Korrektur. »Korrekt.«

\begin{center}
∞∞∞
\end{center}

Mit Überwindung ließ sich Fnbsrd davon überzeugen, dass man sich in die Situation der Stadtbewohner nicht besser einfühlen könne als auf einer »Party«. Ein besonderes Faible hatte man hier für Zeitreisenden-Feiern, deren Zeitpunkt nach Partyende auf lose herumfliegenden Folien am Stadtboden angekündigt wurde.

»Zeitreisendenfeier in Zimmer 205, Metropolis-Hotel. Vorgestern in zwei Stunden.«

»Großartig«, spottete der Wissenschaftler in Mausform. »Wir sind spät dran.«

Wortlos zog Aqua-SOTI einen uralten irdischen Laptop hervor, klappte diesen auf und tippte die Eckdaten in eine Maske. »Das ist nur \emph{deine} Meinung.«

Dann klickte die Enter-Taste und der Partytrip begann.

\begin{center}
∞∞∞
\end{center}

»Moin«, rief jemand in einer fremden Sprache. So fremd, dass es »Moin« in keiner Variante gab. Und doch verstanden die Gäste jedes Wort. »Dich habe ich noch nie gesehen!«

SoulOfTheInternet murmelte beim Händeschütteln, das dürfe gerne noch so bleiben. »Aq…vädüktennetz.« Der Junge hustete plötzlich. Die Erkältung kam und ging.

»Bitte?«

»Entschuldigung, ich musste–« Niesen! Hatschi. »Ich–« \iathought{Das wird so nichts.} Tschumm!

»Ich heiße jedenfalls Erika. Oooh, die Maus ist süß. Na du! Möchtest du mit einem rosa Luftballon fliegen?«

Die Maus kündigte gedanklich bereits ihren Beratervertrag. Zu gestern. Piepsen statt einer hörbaren Antwort ersparte ihr weitere Antworten.

»Und du heißt jetzt wie?«

»SoulOfTheInternet. Du kannst mich gerne ›Soti‹ nennen.«

Erika lachte. »Soti, du hast einen schönen Akzent. Man könnte fast meinen, du kämest von einem anderen Planeten.«

\begin{center}
∞∞∞
\end{center}









»Sprich eine neunzehnstellige Primzahl, Freund, und tritt ein.«

»Eins. Eins. Eins. Eins. Eins. Eins. Eins. Eins. Eins. Eins. Eins. Eins. Eins. Eins. Eins. Eins. Eins. Eins.«

»Binärzahlen zählen nicht.«

»Diese schon.« Fnbsrd starrte die Tür an. »Diese schon, mein Freund.«

Unwillig beugte sich der Zugang dem düsteren Blick und gab den Weg frei.









<~ Musik-Notizzettel ~>

----

Vor-Finale-Idee:

Fnbsrd ist sonst nicht als mächtiger Angriffszauberer bekannt. Die Szene, in der er den Helixnebel im Wald entstehen lässt, könnte aber folgendermaßen eingeleitet werden:

Die Abenteurergruppe muss einen Wolfsangriff auf ein Elfendorf miterleben und ob der Übermacht, angesichts der schieren Masse an magisch geladenen Wölfen, zunächst hilflos dessen Zerstörung beobachten. Fnbsrd reißt irgendwann die Dose Pfeffer aus dem Rucksack und leert diese in einem Zug. Als er dann sieht, wie einer der Wölfe ein Elfenkind angreift, wird es dunkel auf der Welt und Fnbsrd vernichtet die Zeitebene.

~ ~ ~

Die Nacht flackert den Tagesnebel kaputt, das Bild zerbricht im Sternenschein, ferne Sonnen brennen gleißend durch den Staub. Tiefe Zerstörung frisst sich durch den Äther.

Fnbsrd wird durch eine sichtzerreißende Bildstörung zu einer weißen Maus auf dem Arm seines menschlichen Ebenbilds. Die beiden befinden sich auf einem Planeten im Carinanebel, mitten in einer futuristischen Großstadt zwischen zwei Fahrspuren auf einer Verkehrsinsel. Statt einer Fußgängerampel wird eine Wahrscheinlichkeitsberechnung auf einer Leuchttafel dargestellt: Überlebenswahrscheinlichkeit zum aktuellen Überquerungszeitpunkt, in wildfremden Ziffern und vielleicht einem nichtdezimalen System.

~ ~ ~

Fnbsrd zaubert den Helixnebel in den Wald:
Es wird am hellichten Tag sterndunkel. Die Spirale des Helixnebels umspielt den Himmel wie in einer schrägen Jahrmarktattraktion.

Black Wedding (feat. Rob Halford)
In This Moment
 •
Ritual
 • 2017

(noch nie verwendet!)

----


\begin{center}
∞∞∞
\end{center}

Die Seele des Internets saß an einem handelsüblichen Ostbahnhof unter einem großen Steinbogen und spielte gedankenverloren mit sechs azurblauen Würfeln.

»Du weißt, dass der Zug verunglücken wird«, piepste die Maus auf ihrer Schulter ihr leise ins Ohr. »Ein nicht unerheblicher Teil der gleich einsteigenden Menschen wird die Reise nicht überleben.«

»Steht es mir zu, daran etwas zu ändern?«, flüsterte SoulOfTheInternet deprimiert zurück. »Es ist längst geschehen und wird durch unser Handeln nicht vorab rückgängig zu machen sein.«

»Du kannst doch nicht tatenlos dabei zusehen, wie diese Menschen auf ihr Ende warten«, protestierte Fnbsrd.

SOTI sah ihn erstaunt an. »So kenne ich dich gar nicht.«

»Ich habe beschlossen, dir ›Menschlichkeit‹ beizubringen«, behauptete die Maus mit hörbaren Anführungszeichen.

Das ließ sich der Junge in Schwarz nicht bieten. Er hob die Hand und winkte eine neugierig zu ihm blickende Dame heran. »He, Sie.«

Auf zögernde Schritte folgte zweifelndes Schweigen; ein schief gelegter Kopf blinzelte mit würfelblauen Augen.

»Wie heißen Sie?«

»Werding.« Die Frau biss sich auf die Lippen; die Dunkelheit war ihr nicht geheuer. »Werdinger Alkoholfrei? Haben wir leider nicht mehr. Wie war Ihre Frage?«

»Frau Alkoholfrei, ich würde Sie gerne zu einem Würfelspiel einladen.«

»Ich spiele grundsätzlich nicht.«

»Schade«, meinte SoulOfTheInternet.

Die beiden schwiegen sich eine Weile lang gegenseitig an. »Tragen Sie immer eine Sonnenbrille?«

SOTI blinzelte hinter dem Schwarz.

»Auch nachts?«

»Frau Alkoholfrei, Sie spielen ja doch.« Drei Würfel wanderten in ihre Hand; der freie Platz wandelte sich wie von selbst.

»Wir spielen um eine Antwort.«

Fünf, null, sieben. »Ich habe Sie leider nicht verstanden. Wie war Ihre Frage?«

Das breite Grinsen war dem Gast zuwider, doch sie ließ sich nicht beirren. »Ob Sie stets diese Brille tragen.« Eins.

»Mit einer Eins kommen Sie nicht weit«, behauptete der Depp, der selbst mit Spielwürfeln eine Null erreicht hatte. Fnbsrd starrte ungläubig aus dem Ärmel auf die leere Würfelfläche: Der Punkt war auf die höchste Seite eines anderen Würfels gewandert.

Vier, zwei.

»Sehen Sie?«

»Wenig.« Null, Null, Null.

SOTI schüttelte den Kopf. »Kein Wunder ohne Augen.« Neunzehn. »Ich möchte Ihnen die Frage dennoch beantworten.«

Die Frau in Blau starrte gegen die Gläser, dann ertönte laut ein Pfeifen vom Gleis. Zu erschrocken, um hinterherzulaufen, starrte sie ihrem Zug hinterher; sie starrte auch dann noch, als er einen Baumstamm vom Gleis ein kleines Stückchen auf seiner letzten Reise mitnahm.

Als sie sich umdrehte, war der Junge in Schwarz verschwunden; ein Zettel lag auf dem Tisch:

\iaquote{»Try It Again, Charlie Brown.«}

\begin{center}
∞∞∞
\end{center}







Kann etwas an fehlender Vorsicht scheitern, oder nur am Fehlen der Vorsicht?




--------------

Metapher für Wind: Die Stimme des Waldes / das Lied des Waldes

--------------

Vängä erhob ihr Schwert gegen Fnbsrd.

Fnbsrd blinzelte hinter seiner Sonnenbrille, ohne einen Schritt zurück zu weichen. »Vängä, du weißt nicht, was du tust.«

»Niemand stellt sich zwischen mich und das Ziel meiner Rache.« Sie ging einen Schritt auf ihn zu und richtete die Schwertspitze auf seine Brust.

Noch immer bewegte sich Fnbsrd keinen Millimeter, doch die Brille löste sich vom reglosen Kopf. Zwischen Schwert und Brust fiel der Sichtschutz in den Schnee; Todesschwarz durchstach die Eisluft. Der Schwarzmagier blickte der Schwertträgerin tief, tief in die silbernen Augen. Dann hob er die Hände und richtete beide Zeigefinger auf sein Gegenüber. »Zwischen dir und deinem Begehren liegt die Hölle, wilde Kriegerin, und deine Rachsucht endet vor einer Wand aus frischem Vulkangestein.«

Die Elfe zerbrach vor seinen Augen, ging in die Knie, kühlte die Augen am Boden. Das Silberschwert lag in tagelang anhaltender Stille daneben.

»Du wirst uns helfen, den Magier an einen Ort zu tragen, an dem ihm geholfen werden kann.«

»Muss das sein?«, stöhnte die Elfe. Sie war zumindest nach eigenem Empfinden bereits genug belastet.

»Es soll so.«

--------------

























—

"Earth to earth
Ashes to ashes
Dreams to crumbles
And love to dust"


Made of Lies
Elysion
 •
Someplace Better
 •
2014



---


Shikkoku
Maiko Fujita
 •
Shikkoku
 • 2023

—

Auld Lang Syne
(gemeinfrei / Volkslied)

Ich müsste noch aussuchen, von wem es gespielt werden soll.

—

Skydrome
Antti Martikainen
Synthesia

—

<3

Heart Of Flames (feat. Karliene)
Miracle of Sound


—

Zatellite
Wice
 •
2017 Singles
 •
2018

—

Ascension
Miracle of Sound und Karliene

3:20

https://music.youtube.com/watch?v=UHmM9DpjLEI

—

The Path
Miracle of Sound

—

Alien-Frau im mechanischen Reich:

When You Look at Me
Christina Milian

—

Vängä legt los:

Confident
Demi Lovato

—

Angels
Vicetone und Kat Nestel

—

Ist "War of Change" bereits Teil einer Musikliste gewesen? Klingt jedenfalls leicht beschleunigt nochmal aufregender.

—

Feel Invincible
Skillet

—

E for Extinction, auch als Nightcore-Song

—

Unerwartet positiv und hell, sehr melodisch:

Victory
Thomas Bergersen

—

Deutlich weniger eintönig als ich dachte, und schön:

Ameno
ERA

—

Vorwärts in Hoffnung:

The Last of the Mohicans
Kasia Szubert

—

Göttlicher Aufstieg. Gipfel in Sicht. Sonnenlicht halb von der Spitze verdeckt. Strahlen in Winterluft.

Conquest of Paradise
Vangelis

—

Schwieriger Titel, aber nette Melodie, vermutlich inspiriert vom Volkslied/Gassenhauer "Mutter, der Mann mit dem Koks ist da":

Long Live Death
MONO INC.
 •
Welcome To Hell
 •
2018

—

Last Word Hatsudou
U2 Akiyama
 •
Ankokunougakushuu Shinkirou
 •
2013

—

Sie sieht die Sonne
PUR
 •
Seiltänzertraum
 •
1993

—

Weißt Du nicht
PUR
 •
Zwischen den Welten
 •
2018

—

Beinah
PUR
 •
Zwischen den Welten
 •
2018

—

Hör gut zu
PUR
 •
Seiltänzertraum
 •
1993

—

mmh ~

Stay Young
Oasis
 •
Be Here Now (Deluxe Remastered Edition)
 •
1997

—

Seenot
Unheilig
 •
Grosse Freiheit (Digital Version)
 •
2010

—

Kiss It Better
Aluna
 &
MK
 •
Kiss It Better
 • 2022

—

Action, Verfolgungsjagd, Flucht vor weißen Laserbündeln ("schneidendes Licht"):

Get Free
The Vines
 •
Highly Evolved
 • 2002

—

Arouna
Angelique Kidjo
 •
Djin Djin
 • 2007

—

Whoa. Von Beginn an hören, Überraschung bei 0:30. Techno. Auch bis 1:00 voller Überraschungen.

Born Too Slow
The Crystal Method
 •
Legion of Boom
 • 2020

—

stark eskalierender Breakbeat-Techno-Song:

Shadows
The Prototypes
 &
Lily McKenzie
 •
Ten Thousand Feet & Rising
 • 2020

—

Can't Take My Hands off You
Soultans
 •
Can't Take My Hands off You
 • 1996

—

Vielleicht. Für Amaranthe gar nicht schlecht.

Damnation Flame
Amaranthe
 •
Damnation Flame
 •
2023

—

Mein Freund Rüdi
PUR
 •
Nichts Ohne Grund
 •
1991

—

The Walls Of The World
Katie Melua
 •
Secret Symphony (International Version)
 •
2012

—

Wunderschönes Cover:

Blind and Frozen
Raquel Eugenio
 •
Blind and Frozen
 •
2021


—

This I Promise You
*NSYNC
 •
No Strings Attached
 • 2000

—

Larger Than Life
Backstreet Boys
 •
Millennium
 • 1999

—

Lena
PUR
 •
Nichts Ohne Grund
 • 1991

—

Rainbow In The Sky (K & A's Radio Blast)
DJ Paul Elstak
 •
The Best Of Paul Elstak Top 20
 • 2011

—

Skydrome
Antti Martikainen
 •
Synthesia
 • 2016

—

Time Will Catch Me First
Peter Crowley
 •
Collection 9
 • 2020

—

Oh!

I Think I Love You
Voice Of The Beehive
 •
Honey Lingers
 • 1991

—

The Trail
Marcin Przybyłowicz
 •
The Witcher 3: Wild Hunt
 • 2015

—

Overprotected (Radio Edit)
Britney Spears
 •
Britney (Digital Deluxe Version)
 • 2001

—

Torn
Natalie Imbruglia
 •
Left Of The Middle
 • 1997

—

Sick and Tired
Anastacia
 •
Anastacia
 • 2001

—

Time’s a Fantasy (Anyma Remix) (feat. Jeff Bhasker)
Portugal. The Man
 •
Time’s a Fantasy (Anyma Remix)
 •
2023

—

Weird!

Vois sur ton chemin (Techno Mix)
BENNETT
 •
Vois sur ton chemin (Techno Mix)
 •
2023

—

The Scent Of Dead Leaves
Rage Of Light
 •
The Scent Of Dead Leaves
 •
2023

—

Gegen
Silbermond
 •
Himmel auf
 •
2012

—

Unter Feuer
Unheilig
 •
Grosse Freiheit (Digital Version)
 •
2010

—

Tief, düster, schnell pulsierend, melodisch, besonders ab 1:55:

So High
CHROM
 •
Synthetic Movement
 •
2012

—

Arg.

Erlöse Mich
Blutengel
 •
The Victory of Light
 •
2021


—

Fnbsrd-passend

Darkness Awaits Us
Blutengel
 •
The Victory of Light
 •
2021


—

Sprung zu 1:40, Solo beginnt bei 1:45, endet bei 2:00

Lit Up
Buckcherry
 •
Buckcherry
 •
1999



—

Somewhere Better
Ad Infinitum
 •
Somewhere Better
 •
2022


—

Abwärts
Unheilig
 •
Grosse Freiheit (Digital Version)
 •
2010

—

Wiedersehen Woanders
MONO INC.
 •
Ravenblack
 •
2023

—

Sehr schönes Cover, bei Instrumentenwahl und Gesang:

Toss A Coin To Your Witcher (Rock Cover)
Violet Orlandi
 •
9.3M views
 •
237K likes

—

Ou-ha.

Ich liebe dieses Cover. Musikstil und Stimmen passen perfekt dazu.

++

You Spin Me Round (Like A Record) (Radio Edit)
Marc Korn
 •
You Spin Me Round (Like A Record)
 •
2020

—

Now You're Gone [Video Edit] (feat. DJ Mental Theo's Bazzheadz)
Basshunter
 •
Now You're Gone - The Album
 •
2008


—

Cover (wow, tiefe Stimme):

Chasing Cars
Grace Gaustad
 •
Chasing Cars
 •
2018

—

Cover:

Toss a Coin to Your Witcher
Rachel Hardy
 •
Toss a Coin to Your Witcher
 •
2020


—

Ladytron ist zurück auf der Musikliste!
~beautiful~

Mirage
Ladytron
 •
Gravity The Seducer
 •
2011

—

Ladytron ist zurück auf der Musikliste!
~powerful~

I'm Not Scared
Ladytron
 •
Velocifero
 •
2008

—

Thumbs
Sabrina Carpenter
 •
EVOLution
 •
2016

—

Don't Stop Movin'
S Club
 •
Sunshine
 •
2001

—

Wunderschön, lief im Radio:

Chasing Cars
Snow Patrol

—

Nanu, Unheilig hat die Sisters of Mercy gecovert:

This Corrosion
Unheilig
 •
Maschine
 • 2003

Ganz okay.

—

Der Nachfolger von "The Angels Weep" aus den IA1.
Kalte Stadt in Asche.
Zerbrochene Fenster, die schief in ihren Scharnieren hängen. Scherben auf dem Boden.
Neuer Schnee deckt die Apokalypse sanft zu. Der vorherige ist geschmolzen.

Tu Ba Mani
Meghan Kabir
 •
Tu Ba Mani
 • 2015

—

Science Fiction / Zukunftsstadt-Intermezzo:

Zeitmaschinengewehr:

TUCA DONKA
CURSEDEVIL
,
DJ FKU
,
Skorde
 &
MC Roba Cena
 •
TUCA DONKA
 • 2023

—

Science Fiction / Zukunftsstadt-Intermezzo:

Zwischen Hochhäusern:

High
Sir Sly
 •
Don't You Worry, Honey
 • 2017

—

Science Fiction / Zukunftsstadt-Intermezzo:

Roboter:

Put Me to Work
Big Data
 •
Put Me to Work
 • 2019

—

Kaunaz Dagaz
Brothers of Metal
 •
Emblas Saga
 • 2020

—

Out In The Fields
Gary Moore
 &
Phil Lynott
 •
Run For Cover
 •
1985

—

Cheap Thrills
Sia
 •
Cheap Thrills
 •
2015


—

Original?
Jup, sieht so aus
https://en.wikipedia.org/wiki/Unstoppable_(Sia_song)

sehr schön

Unstoppable
Sia
 •
Unstoppable
 •
2016


—

"Remember the feelings, remember the day"

2006er Klassiker mit wunderschöner, eingängiger Melodie.

Bad Boy
Cascada
 •
Everytime We Touch (Premium Edition)
 •
2010

—

Einstieg bei 2:17:

Gimme Shelter
The Sisters Of Mercy
 •
Some Girls Wander by Mistake
 •
1985

—
(übrigens das Lied, das am Anfang von Blutengels "Sing" zitiert wird:)

This Corrosion (New Version for Digital)
The Sisters Of Mercy
 •
Floodland
 •
1985


—

It's Just a Tease
Thundermother
 •
It's Just a Tease
 •
2015

—

Typisch Garbage:

The One
Garbage
 •
Not Your Kind of People (Deluxe Version)
 •
2012


—

Badway
Nina Gordon
 •
Tonight And The Rest Of My Life (DMD Album)
 •
2000

—

Das sind doch wohl Antworten aufeinander:

Find You're Here
Wolfsheim
 •
Find You're Here
 •
2003


Find You're Gone
Wolfsheim
 •
Casting Shadows
 •
2003

—

Hab ich doch bestimmt schon in der Liste:

Typisches Halocene-Cover, alles dabei. Beeindruckend, wie tief die Stimme teilweise geht.

Kickstart My Heart
Halocene
 •
12M views
 •
153K likes


—

Was war zuerst da: Der Star-Control-II-/Ur-Quan-Masters-Soundtrack für "Hyperspace" oder dieses Lied? ;)

Boah, sind die ähnlich. Bis in Minute 1 plötzlich auch noch hinein.

SCII scheint älter zu sein. Vielleicht wurden auch einfach beide vom gleichen Song inspiriert.

Once In A Lifetime
Wolfsheim
 •
Once In A Lifetime
 •
1998


—

Kein Zurück
Wolfsheim
 •
Casting Shadows
 •
2003

—

Ruhige See

oder/und die Ruhe vor dem Sturm

My Jolly Sailor Bold (feat. Colm McGuinness)
Rachel Hardy
 •
My Jolly Sailor Bold
 •
2021


—

Schönes Cover. Ist das Original schon in einer Musikliste gewesen?

Toss a Coin to Your Witcher (feat. Black Gryph0n)
Samuel Kim
 •
Toss a Coin to Your Witcher (feat. Black Gryph0n)
 •
2019



—

Müsste ohne Nightcore bereits drin sein
Meet me on the battlefield


Nightcore - Battlefield
NightcoreReality
 •
62M views
 •
485K likes


—

Nightcore - Yume To Hazakura (Lyrics) 「 Japanese Music 」
The Soul of Wind
 •
9.4M views
 •
124K likes



—

?????

[YUKI] Hakkenden Touhou Hakken Ibun - Yuki no hitohira [雪のひとひら]
Hime no Ashita
 •
7M views
 •
85K likes


—

Laughing In The Sugar Bowl
Veruca Salt
 •
Ghost Notes
 •
2015

—

Verführung des Bösen (König zu Fnbsrd: "Verschone mich und herrsche mit mir")

Reich Mir Die Hand
Blutengel
 •
Reich Mir Die Hand
 •
2011

—

Wonderland
Solitary Experiments
 •
Wonderland
 •
2022

—

Wake Up
Smash Into Pieces
 •
Arcadia
 •
2020

—

Ägyptische Beats:

Flow
Smash Into Pieces
 •
Flow
 •
2023

—

Let Me Be Your Superhero
Smash Into Pieces
 •
Rise and Shine
 •
2017

—

Venom
Smash Into Pieces
 •
Venom
 •
2023

—

All Eyes on You
Smash Into Pieces
 •
Arcadia
 •
2020

—

Das Jahr ist gerade 8 Tage alt. Krass.

Trigger
Smash Into Pieces
 •
Trigger
 •
2024

—

Aha, da haben sich aber einige etwas von abgeguckt. In This Moment mit "Black Wedding" ganz besonders. :)

Ein gleichnamiges Cover von Blutengel existiert und ist auch gut.

White Wedding (Parts 1 And 2) (Shotgun Mix / 24-Bit Digitally Remastered 2001)
Billy Idol
 •
Vital Idol
 • 1987

—

vermutlich schon drauf, aber hier nochmal, Sprung zu 0:35!

Into The Night
Ad Infinitum
 •
Chapter 2: Legacy
 • 2021

—

Learn To Live
Alice Merton

—

Neu, von Alice Merton und sehr gut – seltene Kombination:

Run Away Girl
Alice Merton

—

Meine Welt
Peter Heppner

—

Oh. Stark, tief.

Was bleibt?
Peter Heppner und Joachim Witt

—

Purgatory
Assemblage 23
 •
Contempt
 •
1999

—

Wie ich
Kraftklub
 •
In Schwarz (Deluxe)
 •
2014

—

Alles wegen Dir
Kraftklub
 •
In Schwarz (Deluxe)
 •
2014

—

~ ooooh ~ Ein schöner Kontrast zu "Hell to the Heavens" ~

Fading Earth
Leaves' Eyes
 •
Symphonies Of The Night
 •
2013


—

Keine Angst
Farin Urlaub Racing Team
Faszination Weltraum

—

Das passt doch überall:

AWG (Alles wird gut)
Farin Urlaub Racing Team
Faszination Weltraum

—

Schwer in Fnbsrd zu nutzen, oder?

iDisco
Farin Urlaub Racing Team
Faszination Weltraum

—

Sprung zu Minute 1:00 und zu 2:20

Das Meer
Unheilig
 •
Grosse Freiheit (Digital Version)
 •
2010


—

Lyrics?


Punk fu! (Live)
Daria Zawiałow
 •
Męskie granie 2019
 •
2019


—

ytdl Never Forever (Demo Version)
Elysion
 •
Killing My Dreams
 •
2012

https://music.youtube.com/watch?v=mn3_YMJ_KnQ

—

恋に落ちて (Koi ni Ochite, Fall in Love) – 藤田麻衣子 (Maiko Fujita)

—

Rock Ain't Dead
ONLAP
 •
Rock Ain't Dead
 •
2020

—

"We live for this and nothing more / We are what you created"

E For Extinction · Thousand Foot Krutch

(auch als Nightcore gut)

—

Nur als Nightcore gut: "Can't Go To Hell"
CKuNLoiNx
 •
1.1M views
 •
12K likes

Original:
Can't Go to Hell
Song
 •
Sin Shake Sin
 •
Lunatics and Slaves
 •
3:28

—

Breakout (feat. Scandroid)

celldweller
253K subscribers

https://www.youtube.com/watch?v=uvySKS8TOMM
Nightcore-Version: https://www.youtube.com/watch?v=IDwq_xEmikk

—

https://en.wikipedia.org/wiki/Yuki_no_Hana

Gutes Cover:

雪の華 (Yuki no hana) - 中島美嘉 // covered by 凪原涼菜
Suzuna Nagihara
 •
2M views
 •
24K likes


Quelle (deutlich ruhiger):

Yuki No Hana
Mika Nakashima
 •
Yukinohana
 •
2003

—

I Left My License in the Future
Thundermother
 •
I Left My License in the Future
 •
2023

—

Dog from Hell
Thundermother
 •
Heat Wave
 • 2020

—

Heat Wave
Thundermother
 •
Heat Wave
 • 2020

—

Comatose
Skillet
 •
Comatose
 • 2006

—

(Ebenfalls als Nightcore-Version besonders ohrwurmlastig:)

Confident
Demi Lovato
 •
Confident
 • 2015

—

Meet Me on the Battlefield
Svrcina
 •
Meet Me on the Battlefield
 • 2016

—

Irgendwie, Irgendwo, Irgendwann (feat. Denis Sarp, Michelle Barth, Tobias Görtzen & Pablo Ludwig)
Das Cover der Band "Kontrollverlust" ist sehr, sehr gut.

Das wäre auch einer der wenigen Fälle, in denen ich bewusst eine Coverversion in der Musikliste aufführen würde.
 •
Abgestaubt
 • 2020

—

Klingt beschleunigt (z.B. als "Nightcore"-Version) schöner.
Erinnert mich stellenweise (0:55-1:15) an die Melodie von "Irgendwie, irgendwo, irgendwann":

Broken Angel - Feat. Helena (feat. Helena)
Arash
 •
Superman
 • 2014

—

Ein ungewöhnlich positiver, organisches Leben bejahender Track von Assemblage 23:

Binary
Assemblage 23
 •
Meta
 •
2007

—

Starker Start, Metropolenpuls.

Außerdem ein äußerst Fnbsrd-passender, geradezu prophetischer Text:

Ghosts
Assemblage 23
 •
Meta
 •
2007

—

okay:

Sorry
Assemblage 23
 •
Meta
 •
2007

—

Das ist schon ein übler Sägezahn im Intro:

Raw
Assemblage 23
 •
Meta
 •
2007


—

Warum in alles in der Welt ist Hasselhoff dabei …

… und warum klingt das dann auch noch gut?

Through the Night (feat. David Hasselhoff)
CueStack
 •
Through the Night
 •
2020

—

Stellenweise schön:

The Devil Went Down to Georgia
State of Mine
 &
The Family Tradition Band
 •
The Devil Went Down to Georgia

—

Regret
Assemblage 23
 •
Storm
 •
2004

—

Oh.
Warum, falls das noch nicht in der Liste ist, ist es dort noch nicht drin?

Und nun habe ich endlich die SRB2-Melodie gefunden. War ja klar, dass das kein Original war.

Ui ui ui

Kiss from a Rose
Seal
 •
Seal
 •
1994


Und, Cover:

Kiss from a Rose
No Resolve
 &
KAYLA KING
 •
Kiss from a Rose
 •
2022

—

Unstoppable
Ad Infinitum
 •
Chapter 2: Legacy
 • 202

—

What We're Up Against feat. Elize Ryd of Amaranthe (feat. Elize Ryd, Amaranthe & Elize Ryd of Amaranthe)
Apocalyptica
 •
What We're Up Against
 •
2023

—

Whatever
Thundermother
 •
Thundermother
 •
2018

—

Vielleicht, vielleicht … für die Zukunftsstadt …

Digital World
Amaranthe
 •
MASSIVE ADDICTIVE
 •
2014

—

Introkribbeln, hochwertig produziert, tiefe Basspräsenz.
Der Liedtext kündigt Rache an, was durch die tiefe ruhige, positive, nach oben strebende Melodie einen recht merkwürdigen Eindruck verströmt. Es könnte aber zu Fnbsrd passen.

For All We Have To Suffer
MONO INC.
 •
Symphonies of Pain - Hits and Rarities
 •
2017

—

Vermutlich bereits in der Liste.

Lieb' Mich
MONO INC.
 •
Ravenblack
 •
2023

—

Bereits in der Liste, aber einfach nochmal:

Princess Of The Night
MONO INC.
 •
Ravenblack
 •
2023

—

STAR WALKIN' (League of Legends Worlds Anthem)
KAYLA KING & Lauren Babic
 •
STAR WALKIN' (League of Legends Wo

—

Until Forever
Eva Under Fire
 •
Anchors
 •
2015

—

Face Down
Halocene
 &
Lauren Babic
 •
We've Got It Covered: Vol 7
 •
2020

—

1987
DIAMANTE
 •
1987
 •
2023

—

American Dream
DIAMANTE
 •
American Dream
 •
2021

—

Goldene Melodie:

Haunted
Ad Infinitum
 •
Chapter 2: Legacy
 •
2021

—

Ballade:

Breathe
Ad Infinitum
 •
Chapter 2: Legacy
 •
2021

—

Afterlife (feat. Nils Molin)
Ad Infinitum
 •
Chapter 2: Legacy
 •
2021

—

Melissa Bonny ist eine Wucht:

Into The Night
Ad Infinitum
 •
Chapter 2: Legacy
 •
2021

—

Oh, sehr gutes Cover des Billie-Ellish-Songs (springe z.B. zu 2:00):

Bad Guy
Conquer Divide
 •
Bad Guy
 •
2020

—

Jennifer Haben :) <3

Halo Of The Dark
Beyond The Black
 •
Lost In Forever
 •
2016

—

unerwartet stark:

Lost Inside
Nemesea
 •
In Control
 •
2007

—

Energize Me
After Forever
 •
After Forever
 •
2007

—

Oha, dunkel, düster, hart, und das von Nightwish:

Dark Chest of Wonders
Nightwish
 •
Once
 •
2004

—

Schöner Klassiker, muss man sich drauf einlassen:


Ice Queen (Radio Edit)
Within Temptation
 •
Ice Queen (EP)
 •
2001

—

Hübsch. Die Melodie ist aber nicht von denen?

Ha. Das könnte "New Divide" von Linkin Park sein.

Dim Days Of Dolor
Sirenia
 •
Dim Days Of Dolor
 •
2016

—

Lyrics passen zu Fnbsrd als Fantasywelt:

The Gathering
Delain
 •
Lucidity
 •
2006

—

Lyrics passen zu Fnbsrd als Person:

Dreamer
Elysion
 •
Silent Scream
 •
2009

—

Vermutlich als Kriegs- und/oder Religionskritik gedacht, aber nicht so klar, wie man sich das vielleicht wünscht.

The Last Crusade
MONO INC.
 •
The Book Of Fire
 •
2020

—

I Want Out
Danko Jones
 •
Power Trio
 • 2021

—

Guess Who's Back
Danko Jones
 •
Electric Sounds
 • 2023

—

Diese göttlichen Melodien.

Und dazu:

"Every day takes me further away
From this sadness of mine
And every day takes me further away
At the end of the rainbow I'll shine again"

At The End Of The Rainbow
MONO INC.
 •
Ravenblack
 • 2023

—

Into the Mud
Thundermother
 •
Heat Wave (Deluxe Edition)
 • 2021

—

Like A Phoenix
The Gems
 •
Like A Phoenix
 • 2023

—

Hübsch :)

Louder Than Hell
MONO INC.
 •
The Book Of Fire
 •
2020


—

Wundervoll.

Ice Queen (Radio Edit)
Within Temptation
 •
Ice Queen (EP)
 •
2001


—

Far Away
Elysion
 •
Bring Out Your Dead
 •
2023


—

Ghostship
MONO INC.
 •
Symphonies of Pain - Hits and Rarities
 •
2017


—

Schon Teil der Fnbsrd-1-Musikliste gewesen? Falls nicht:

Sleepwalking
The Birthday Massacre
 •
Pins And Needles
 •
2010

—

Loss
Elysion
 •
Silent Scream
 •
2009


—

Black Diamond [Live]
Stratovarius
 •
Visions of Europe (Reissue 2016) [Live]
 •
2016


—

Paid in Full
Sonata Arctica
 •
Unia
 •
2007


—

"Heart of an eagle, he flies through the rainbow
Into a new world and finds the sun"

Eagleheart
Stratovarius
 •
Elements Part 1
 •
2003

—

Weballergy
Sonata Arctica
 •
Silence
 •
2002

—

San Sebastian (Revisited)
Sonata Arctica
 •
Silence
 •
2002


—

Leiser und ruhiger als andere Lieder:

Serves You Right
DIAMANTE
 •
Serves You Right
 •
2020


—

Der starke Fokus auf Augenfarben bei Fnbsrd erfordert ja fast, dass dieser Song eingebaut wird – besonders mit Blick auf das Single-Cover, das eine blau-rot gefärbte Iris darstellt:

Iris
DIAMANTE
 &
Breaking Benjamin
 •
Iris
 •
2020


—

Ruhig, nachdenklich, episch; Sprung zu 2:15 für einen Melodiewechsel innerhalb von 15 Sekunden.
"Take me under the winds that blow"

Freedom
Visions of Atlantis
 •
Pirates
 •
2022


—

Fnbsrd 3:

Legion Of The Seas
Visions of Atlantis
 •
Melancholy Angel
 •
2022


—

Sehr schön, aber der Text ist konfrontativ:

Melancholy Angel
Visions of Atlantis
 •
Melancholy Angel
 •
2022

—

Melancholy Angel (Orchestral Version)
Visions of Atlantis
 •
A Pirate's Symphony
 •
2023


—

Wunderschön geworden, und zudem sogar Fnbsrd-passend:

Survival Anthem
URTONMUSIC
 •
7 vs. Wild: Teams - Season 3 (Original Series Soundtrack)
 •
2


—

E Nomine--Schwarze Sonne
Ben Dover   ← warum?
 •
2.5M views
 •
17K likes


—

Bei 1:10!

The Gathering
Delain
 •
Lucidity
 •
2006


—

Fnbsrd schiebt genervt einige Gegner beiseite:

Heute nicht!
Ben Zucker
 •
Heute nicht!
 •
2023



—


Sehr ruhige Melodie:

Nemesis
MONO INC.
 •
The Book Of Fire
 •
2020


—

Welcome To Hell
MONO INC.
 •
Welcome To Hell
 •
2018



—

Jesus Christus, das Intro

Wenn das nicht auf der Musikliste landet, stimmt etwas nicht

Arabia
MONO INC.
 •
After The War
 •
2012




—

Nett :)

Forever Yours
Sunrise Avenue
 •
On The Way To Wonderland
 •
2006


—

Ui. Wie jetzt… Dass es das gibt…

Schenk mir Dein Herz
Down Below
 •
Mutter Sturm
 •
2015


—

Hochhäuser durch rote Wolken, schwarzer Asphalt:

Feuerland
Unheilig
 •
Lichter der Stadt
 •
2012



—

Tage wie Gold
Unheilig
 •
Lichter der Stadt
 •
2012

—

Der Anfang ist musikalisch etwas schwach, aber das wird gut:

"Now speak to me, I trust in you
Your vision's clear, your words are true
Your eyes reveal the world behind…"

Speak To Me
Rotersand
 •
Random Is Resistance
 •
2005


—-

Beat And The Pulse
Austra
 •
Feel It Break (Deluxe Version)
 •
2011


—

Die Noten Deines Lebens
MONO INC.
 •
Terlingua
 •
2015


—

Threefold Law
Nemesea
 •
Mana
 •
2004


—

Wunder-schrägschönes Intro, starke Gitarren im Hintergrund, gute Instrumentwahl, hübsche Melodie mit Überraschungen:

Voices Of Doom
MONO INC.
 •
Symphonies of Pain - Hits and Rarities
 •
2017



—

River Flows in You
Yiruma
 •
SOLO: 20th Anniversary
 •
2021


—

Erinnerungen…

Eisblumen (Radio Mix)
Eisblume
 •
Unter dem Eis (Re Release)
 •
2008


—

I Am the Fire
Halestorm


—

1987
DIAMANTE
 •
1987
 • 2023

—

Idee für Aussage im »Try walking in my shoes«-Stil:

»Du liegst so tief unter meinem Niveau –«, laberte der König, doch er wurde von Fnbsrd unterbrochen:

»Komm zu mir nach unten. Wir suhlen uns im Dreck und ich gehe sauberer wieder heraus, als du hineingegangen bist.«

Dann ergriff Fnbsrd die Krone und schmiss sie so kräftig neben sich, dass brechende Zacken unter dem Metall davonschlitterten.

»Hol dir dein ›Niveau‹ am besten mit einem Handfeger. Es ist nicht mehr viel davon übrig.«

—

E Nomine:
"Vater Unser" bei Lichtmagie
"Mitternacht" bei Dunkelmagie
Möglicherweise "Der Herr der Finsternis" für Fnbsrd

—

Lieb Mich
Joachim Witt

—

Sing
Blutengel
 •
Sing
 •
2015



—

Ma Version (Bright White Lightning vs Dubmood)
Bright White Lightning
 •
Dirty Nails
 •
2014


—

Stupefied (feat. Bright White Lightning)
Dubmood
 •
Machine
 •
2014


—

Fremde Großstadt: (sehr schön, Willkommen zurück in meinen Musiklisten, Dubmood)

Silver Screen [Part 3] (feat. Bright White Lightning)
Dubmood
 •
Overshoot Days
 •
2018

—

Das Coverbild von rotersand's "Random is Resistance"-Album ist großartig, das könnten yury, Orakel und Free sein.

Damit müsste man doch irgendetwas machen können.

—

Disguise (feat. Culture Kultür)
Mental Discipline
 •
Constellation
 •
2012



—

Butterfly
Mental Discipline
 •
Past Forward
 •
2018


—

My Name
Mental Discipline
 •
My Name
 •
2022


—

Rock You Like a Hurricane (feat. Violet Orlandi)
Sershen&Zaritskaya
 •
Rock You Like a Hurricane
 •
2021


—

Erhebend in Erhabenheit:

Bis ans Ende der Zeit
Joachim Witt
 •
Neumond
 •
2023




—

"Die Erde brennt, ich steh im Feuerland mit dir
(...)
Die Erde brennt, die Sonne wird so schwarz wie Teer"

Die Erde brennt
Joachim Witt
 •
Neumond
 •
2023


—

wtf, das ist fast Pop

Propaganda
Joachim Witt
 •
Der Fels in der Brandung
 •
2023



—

Die Zukunftsstadt in der Vergangenheit (Carinanebel):
(dafür dann sehr passend!)

Quo Vadis (feat. U96)
Joachim Witt
 •
Rübezahl
 •
2018




—



Mein Diamant
Joachim Witt
 •
Rübezahl
 •
2018





—-

Dämon
Joachim Witt
 •
Rübezahl
 •
2018





—

Oh!

Fnbsrd-2-Titel?

Herr der Berge
Joachim Witt
 •
Rübezahl
 •
2018




—


Nuclear Jäger (Vs. Hisoutensoku Mk. II) - Touhou: Scarlet Curiosity OST | Hachimitsu-Lemon
Pcah
 •
598 views
 •
15 likes



—


Simple Minded, Bothersome and Great Youkai (Vs. Suika) - Touhou Scarlet Curiosity | Hachimitsu-Lemon
Pcah
 •
376 views
 •
6 likes




—




Laissez-faire Youkai Castle (Palace of Earth Spirits) - Touhou: Scarlet Curiosity | Hachimitsu-Lemon
Pcah
 •
302 views
 •
11 likes


—


Adventures of Scarlet Curiosity (Title) - Touhou: Scarlet Curiosity OST | Hachimitsu-Lemon
Pcah
 •
913 views
 •
9 likes




—-


Midnight Curio Shop (Kourindou: Night) - Touhou: Scarlet Curiosity OST | Hachimitsu-Lemon
Pcah
 •
489 views
 •
12 likes


—

Stark und tief weiblich gesungen, anspruchsvolle Melodie, aber mit einem nur sehr schwierig verwendbaren Titel:

Parasite
Anette Olzon
 •
Strong
 • 2021

—

My Favourite Stranger
Depeche Mode
 •
Memento Mori
 • 2023

—

Achtung, nicht einfach von ZUN wie frühere Touhou-Soundtracks:

Hexagonal Stone Cage (Genbu Ravine) - Touhou: Scarlet Curiosity OST | Hachimitsu-Lemon

—

Starke Melodie:

Princess Of The Night
MONO INC.
 •
Ravenblack
 • 2023

—

Springe zu 1:20:

Last Ride Of The Day
Nightwish
 •
Imaginaerum
 • 2011

—

Gespenstisch; kräftiges Intro; schrägschöne Melodien:

The Siren
Nightwish
 •
Once
 • 2004

—

Children of the Dark (feat. Chris Harms, Tilo Wolff & Joachim Witt)
MONO INC.

—

Höchst interessante Melodie mit genialem Gitarrenspiel ab 1:51:

Black River Delta
Visions Of Atlantis
 •
Delta
 • 2011

—

The Underworld
Ad Infinitum
 •
Chapter III - Downfall
 • 2023

—

Zerstörung des (Prä-)Mechanischen Reichs:

Upside Down
Ad Infinitum
 •
Chapter III - Downfall
 • 2023

—

Eternal Rains
Ad Infinitum
 •
Chapter III - Downfall
 • 2023

—

Vertigo
U2
 •
How To Dismantle An Atomic Bomb
 • 2004

—

Little By Little
Oasis
 •
Heathen Chemistry
 • 2002

—

Legacy – In This Moment

—

Schenk mir ein Wunder
Unheilig
 •
Grosse Freiheit (Digital Version)
 • 2010

—

So wie Du warst
Unheilig
 •
Lichter der Stadt (Deluxe Edt.)
 •
2012


—

A Ghost of a Chance
The Murder Of My Sweet
 •
A Ghost of a Chance
 •
2021

—

Unbreakable
The Murder Of My Sweet
 •
Unbreakable
 •
2012

—

Fallen
The Murder Of My Sweet
 •
Bye Bye Lullaby
 •
2012


—

Mit "Girl" statt "Man" und Klaviermusik, schöne Idee:

Behind Blue Eyes
Within Temptation
 •
The Q-Music Sessions
 •
201


—

Schön schräg:

In This Together
Apoptygma Berzerk
 •
You And Me Against The World
 •
2005


—

Klingt wie SC2/UQM:
Diamonds
The Birthday Massacre
 •
Diamonds
 •
2020


—-

Oha. Neuer Unheilig-Favorit.

Sternbild
Unheilig
 •
Grosse Freiheit (Digital Version)
 •
2010


—

Sleeping My Day Away
MONO INC.
 •
Pain, Love & Poetry (Collector's Cut)
 •
2013


—

Schön gestaltet:

Vampire
Blutengel
 •
Un:Gott
 •
2019


—

Okay, Zeit das Album zu kaufen:

Rome Wasn't Built In A Day
MONO INC.
 •
Together Till The End
 •
2017

—-

Schönes Cover:
(war von Sharon Den Adel aber auch kaum anders zu erwarten)

Summertime Sadness
Within Temptation
 •
Hydra (Deluxe Edition)
 •
2014



—

Acid Ocean
Project Pitchfork
 •
Black
 •
2013

—

Forever And A Day
MONO INC.
 •
Together Till The End
 •
2017


—

"Where the rainbow starts to end…"
"Far far away…"

Boatman (Live in Hamburg)
Mono Inc.
 •
Boatman (Live in Hamburg)
 •
2023

—--

I Close My Eyes
Covenant
 •
The Blinding Dark
 •
2016


—--

Waiting To Be Born
Rotersand
 •
Random Is Resistance
 •
2005



—-

Das war noch nicht auf der Liste?

Drive
Assemblage 23
 •
Defiance
 •
2002


----


Set Fire to the Rain – Adele


----

Heartbreaker
Pat Benatar
 •
Greatest Hits
 • 2005

----


Shine And Shade
Beyond The Black
 •
Lost In Forever
 • 2016

 ---


The Truth Beneath the Rose
Within Temptation
 •
The Heart Of Everything
 • 2007

 ----



Stronger
Sugababes
 •
Angels With Dirty Faces
 • 2002




—---


Our Solemn Hour
Within Temptation
 •
The Heart Of Everything
 •
2007




—-




What Else Is There? (feat. Zora Cock)
Mortemia
 &
Sirenia
 •
What Else Is There?
 • 2022


—-




Deadlight
Sirenia
 •
Deadlight
 • 2023


—-------




Wintry Heart
Sirenia
 •
Wintry Heart
 • 2023




—--












Written In Blood Beyond The Black • Lost In Forever • 2016
…








Verwundeter Sanitäter: “Wounded Healer” von Beyond the Black und Elize Ryd
(wird bei der Verteidigung des Silberdrachen koroljow verletzt)


—--








“Metfest” von Feuerschwanz: Elfendorf?
Jedenfalls rein!




—---------------














Screenshots vom 2023-05-16 voller Musikerkennungen, die sich wirklich lohnen.
Summer Son allen voran.




—--


"Stupefied" von Dubmood und Bright White Lightning ist so dermaßen gut, dass es rein muss, zum Beispiel gegen Romanende. Lyrics enthalten: "But you're ugly now", "high as a kite last night", "so please don't dope".


—-




The Gates Of Time
The Dark Side of the Moon
 •
The Gates Of Time
 • 2023










—--


MAGNUS frisst Lebensenergie:
“Unlimited Sin - Unlimited Power!”, FROM HELL WITH LOVE!


Unlimited Sin
Beast In Black
 •
From Hell with Love
 •
2019








—--


Legends Never Die
The Dark Side of the Moon
 •
Legends Never Die
 •
2023


oder das Original:


Legends Never Die (feat. Against The Current)
League of Legends
 •
Legends Never Die
 •
2017






—-
This Corrosion (New Version for Digital)
The Sisters Of Mercy
 •
Floodland
 •
1985


—


Legends (feat. Chrigel Glanzmann)
Ad Infinitum
 •
Chapter III - Downfall
 •
2023


—---


Feindlicher Verführungsversuch gegenüber… Vängä, nehm ich mal an:
(Lyrics)


Lucretia My Reflection (Vinyl Version) [New Version for Digital]
The Sisters Of Mercy
 •
Floodland
 •
1985


–


Love in a Trashcan
The …ettes


—-










Neuer Titelsong?
Sehr beeindruckend.

San Diego (The Tragical)
The Eternal Afflict
 •
Re(a)lict Or Requiem
 •
2008



—


“Don’t walk away when the world is burning”: FNBSRD 2


Carnival of Rust
Poets of the Fall
•
Carnival of Rust
• 2006
Carnival of Rust
Poets of the Fall
 •
Carnival of Rust
 •
2006


—-


Beautiful Delirium
Blackbriar
 •
Our Mortal Remains
 •
2019








—-----


Bye Bye Beautiful
Nightwish
 •
Dark Passion Play
 •
2007




—--




Sea of Fire
Nocturna
 •
Daughters of the Night
 •
2022




—--






Our Solemn Hour
Within Temptation
 •
The Heart Of Everything
 •
2007




—-------
wow
Lost In Life
Sirenia
•
The 13th Floor
• 2009






—-
Angel Cry (MFiT)
Battle Beast
 •
Unholy Savior
 •
2015








—-






Lovesong
The Cure
 •
Galore - The Singles 1987-1997
 •
1997




—


Every You, Every Me
Placebo
 •
A Place For Us To Dream
 •
2016




—














“Vision Thing” - Sisters of Mercy
Something Fast
The Sisters Of Mercy















https://music.youtube.com/channel/UCI6B8NkZKqlFWoiC_xE-hzA
YOASOBI
Yoasobi is a Japanese music superduo formed by Sony Music Entertainment Japan, composed of Vocaloid producer Ayase, and singer-songwriter Ikura. Represented by the slogan "novel into music", the duo has released songs based on novels posted on Monogatary.com, a novel-centered social media operated by their label, which is also from various media like novels written by professional authors, books, letters, and plays.


夜に駆ける
YOASOBI  •  THE BOOK  •  2021
https://music.youtube.com/watch?v=mJ1N7-HyH1A


群青
YOASOBI  •  THE BOOK  •  2021
https://music.youtube.com/watch?v=dGZqpVCJP3k


セブンティーン
YOASOBI  •  セブンティーン  •  2023
https://music.youtube.com/watch?v=q-D2VzYC3lw
bis jetzt mein Favorit, weil zwischendurch schön schräg.






— — — — — — —


Mr. SOL - SuperStyle - YouTube Music

Bomfunk MC's - Freestyler (JONNI Remix) - YouTube Music

(beide gesichert)


—---------




Đêm, đom đóm và em - 夜と蛍と私. (Nghe hay hơn khi đeo tai nghe)
Strider Zeno
• 149K views • 2.9K likes
Đêm, đom đóm và em - 夜と蛍と私. (Nghe hay hơn khi đeo tai nghe) - YouTube Music




—


















Vielleicht zu technisch für einen Fantasy-Roman:
* “Io Stresso” (Album Edit) von Dubmood und Gem Tos.
* “Perdition” von Gem Tos und Dubmood. (wirklich wichtig und doch fantasygeeignet)
* “Sei Vecchi” von Gem Tos und Dubmood (Discomusik?)
* “Space is the Place” von Gem Tos und Dubmood (okay-elektronisch)
* “Silver Screen” von Gem Tos und Dubmood (relativ sicher zu elektronisch)
* “Depassement” von Dubmood und Gem Tos. In den IA2 einmal kurz verwendet.


“Lai Lai Hei” von Ensiferum


“Across the Sea” von Leaves’ Eyes


“Making Up Again” von Goldie


“Suffer Well” von Depeche Mode hat einen Daumen nach oben.


“You Know I'm No Good” von Amy Winehouse




https://www.youtube.com/watch?v=MKeTuZ-zoeA
“Living in the City” aus Sonic R.
Musikstil “Dream Pop”
Siehe auch “No More Lies” von Tame Impala und noch irgendwem.


Fnbsrd 3


“Es klingt nach Freiheit” - Santiano
(und weitere Santiano-Lieder. 500 Meilen?)
in einem Fnbsrd-Teil, in dem die Reise übers Meer geht.
curie, koroljow und MAGNUS, Letzterer auf Koroljow, auf großem Flug.
Mit Smaragden kehren sie zurück und helfen Fnbsrd gegen eine neue Bedrohung.








Intro-Kapitel vor dem ersten Teil: “Lady of Worlds” braucht dringend ein Comeback, vielleicht zu Beginn von Fnbsrd 2. Vängä stößt bei der Rückkehr von einem Ausflug auf die Überreste ihres neuen Heimatdorfs. Angenagte Knochen. Aasgeier. Die Wölfe sind zurück.


Und erst mit DIESER Erkenntnis beginnt das Buch.


“Fnbsrd 2: Die Rückkehr der Wölfe”


Kapitel 1: Fnbsrd und Aqua heiraten.







----


U Got 2 Let The Music
Cappella
 •
U Got 2 Know Album
 • 1994

----

Move On Baby
Cappella
 •
U Got 2 Know Album
 • 1994

----

Move It Up
Cappella
 •
U Got 2 Know Album
 • 1994

----

Ready For Love
Cascada
 •
Everytime We Touch
 • 2006

 ----

Hinunter bis auf Eins
Unheilig
 •
Gipfelstürmer
 • 2014

----

Goldrausch
Unheilig
 •
Gipfelstürmer
 • 2014

----

Dem Himmel so nah
Unheilig
 •
Gipfelstürmer
 • 2014

----

With A Smile
Maiah Manser
 •
With A Smile
 • 2023

----

Regret
Assemblage 23
 •
Storm
 •
2004

----

True Faith (Shep Pettibone 12" Remix)
New Order
 •
Brotherhood (Collector's Edition)

----

Falls noch nicht verwendet:

Me And You
Camouflage
 •
Sensor
 •
2003

----

Sehr cool.

Deep Sea Bass
Danny Baranowsky
 •
Crypt of the Necrodancer (Original Game Soundtrack)

----

Separate Ways (Worlds Apart) (Stripped Down) (feat. Lzzy Hale)
Daughtry
 •
Separate Ways (Worlds Apart) (Stripped Down)
 •
2023

("
Someday love will find you
Break those chains that bind you
One night will remind you
How we touched and went our separate ways
If he ever hurts you
True love won't desert you
You know I still love you
Though we touched and went our separate ways
")

----

Heavy Is The Crown
Daughtry
 •
Dearly Beloved
 • 2021

----

Baianá
Bakermat
 •
The Ringmaster
 •
2020

----

Love Again
Dua Lipa
 •
Future Nostalgia
 •
2020

----

Disarm (Alex O. Mix)
Apoptygma Berzerk
 •
Disarm (B-Sides & Rarities)
 •
2020

----

Suffer Well
Depeche Mode
 •
The Best of Depeche Mode, Vol. 1 (Deluxe)
 •
2006

----

Muss rein:

Martyr
Depeche Mode
 •
The Best of Depeche Mode, Vol. 1 (Deluxe)
 •
2006

----

Whatever It Takes – Imagine Dragons

----

Rückfahrt übers Meer, Fnbsrd 3:

Rolling Home
Santiano
 •
Mit den Gezeiten (CH Sonderedition)
 •
2013

----

Das ist eure Zeit
Santiano
 •
Wenn die Kälte kommt
 •
2021

----

Oh wow, wow schön <3


Richtung Freiheit
Santiano
 •
Von Liebe, Tod und Freiheit
 •
2015

----

Apokalypseblick:

Wir sind uns treu
Santiano
 •
Mit den Gezeiten (Special Edition)
 •
2013

----

Die Antwort weiß der Wind
Santiano
 •
Die Antwort weiß der Wind
 •
2022

----

Die Melodie ist hübsch.

Brüder im Herzen
Santiano
 •
Haithabu - Im Auge des Sturms
 •
2018

----

Ouuuh, "Scarborough Fair" ist möglicherweise einen Versuch wert.
Klingt ein bisschen wie das "Lullaby of Deserted Hell" aus Touhou und hat dazu noch Anklang von "House of the Rising Sun" von den Animals. Auf jeden Fall muss das rein.
Hier die Santiano-Version:

Garten Eden (Scarborough Fair)
Santiano
 •
Bis ans Ende der Welt (Second Edition)
 •
2012

----

Nichts als Horizonte
Santiano
 •
Wenn die Kälte kommt
 •
2021

~ oder, dunkler: ~

Nichts als Horizonte (Rock Mix)
Santiano
 •
Nichts als Horizonte (Rock Mix)
 • 2022

----

Separate Ways (Worlds Apart) (Stripped Down) (feat. Lzzy Hale)
Daughtry
 •
Separate Ways (Worlds Apart) (Stripped Down)
 •
2023

----

Sehr gut, Text passend:
Help Is On The Way – Rise Against

Sehr gut:
Re-Education (Through Labor) - Rise Against
Make It Stop (September's Children) - Rise Against

Okay:
Sattelite – Rise Against
Savior – Rise Against

»Facer«~– X-Marks The Pedwalk

----

»No One Left to Lose«~– Iris

----

Übelste, tiefst-sarkastische Auseinandersetzung mit dem Verhalten von Sekten:
»Wunderschön«~– Staubkind

(damals noch eine Dark-Rock-Band; das Lied ist aus ihrem ersten Album)

----

Langer Song (7 Minuten): Ganz nett gemacht. Schweizer Metalband mit Frontsängerin!

»Death«~– Cellar Darling

----

unbekannte Metalband aus Frankreich:

»This Is the End«~– Dust In Mind

----

"Whisper My Child" von Eleine

----

Schön:
"Song zu viel" von Staubkind

"Kämpferherz" von Staubkind und Iris Mareike Steen
 (Lyrics passend zu curie)
Oh, und mit Gastsängerin. Sehr schön.

Vielleicht nicht für Fnbsrd, aber tief:
"Vielleicht irgendwann" von Staubkind
Und dann wieder positiv:
"Dem Himmel so nah" von Staubkind. Nicht schlecht.

----

On Your Own – Meltt

interessantes Albumbild

----

Comeback für:

Oh Lord
In This Moment
 •
Ritual
 • 201

----

Schönes Cover. Merkwürdiges Albumbild und merkwürdige Band. Prüfen.

Listen To Your Heart
Through Fire
 •
All Animal
 • 2019

----

Hübsch:

The Twilight In Your Eyes
Sirenia
 •
The Enigma of Life
 • 2011

----

"The Sky Is Crying" von Ayron Jones ist musikalisch traurig und nicht besonders mitreißend, aber der Text ist nicht schlecht für Fnbsrd:

…
The darker it gets, the more I see
…
'Cause I'm only happy when the sky is crying
And I only smile when I feel the pain
If I'm not alive then what's the point in dying
'Cause I'm only happy in the pouring rain
…

----

Contretemps
Elyose
 	•
Ipso Facto
 • 2015

----

The Dream
The Birthday Massacre
Violet
 • 2004

----

"Chronocide" – Elyose
Fnbsrd verbindet zwei Universen und fügt der Zeit tiefe Schnittwunden zu.

----

Da ist immer noch mein Herz
Album • Staubkind • 2023

Intro; ← sehr ungewöhnlicher Eintrag für die Musikliste. Kann oder muss gar rein.
Da ist immer noch mein Herz; ← sehr schön angespielt

----

wow, ok

Skinny Love
Birdy
 •
Birdy
 • 2011

----

Free
Ayron Jones
 •
Child of the State
 • 2021

----

Hat was von Crush 40, schöner Hardrock:

Strawman
Ayron Jones
 •
Chronicles Of The Kid
 • 2023

----

Baptized in Muddy Waters
Ayron Jones
 •
Child of the State
 • 2021

----

Mercy
Ayron Jones
 •
Child of the State
 • 2021

----

Achtung, Fnbsrd-Geschichte, extrem schilddurchschlagend:

Blood In The Water
Ayron Jones
 •
Blood In The Water
 • 2023

Ich finde, der Song muss unbedingt rein, aber er ist nichts für schwache Gemüter.

----

Fnbsrd, wie er leibt und lebt:
(Klingt wie Nickelback, ist aber neu)

Nobody Praying For Me
Seether
 •
Isolate And Medicate (Deluxe Edition)
 • 2014

----

Frieden.

Still
Daughter
 •
If You Leave
 • 2013

----

Ein seltener Moment von Qualität in der Musikliste.

Words as Weapons
Birdy
 •
Fire Within (Deluxe)
 • 2013

Selten. Nachdenklich. Schön.

----

Weniger qualitativ, aber kein Cover, sondern ein doch recht hart durchschlagender eigener Song:

Words As Weapons
Seether
 •
Isolate And Medicate
 • 2014

----

Auf dem Cover-Trip! Rock/Metal/Electro? Jedenfalls hart!

Counting Stars
No Resolve
 &
Saving Abel
 •
Counting Stars
 • 2023

Das Original ist von OneRepublic und taucht als "Wiedersehensfest im Park" in der IA3-Musikliste auf.

----

Ein schönes Cover, falls ein Comeback des bereits in Vängäs Hütte in Fnbsrd 1 genutzten Lieds gewünscht ist:

Dream On
Blacktop Mojo
 •
Dream On
 • 2016

----

Wer davon ein Hardrock-Cover erstellt hat, gehört in die Musikliste erhoben:

Gangsters Paradise
Like A Storm
 •
Awaken the Fire
 • 2015

Wow, wow, wow

----

Gothic-Industrial Metal:

(hier instrumental, die anderen Songs sind in Französisch geschrieben)

Fragrances
Elyose
 •
Ipso Facto
 • 2015

Hier mit Gesang, sehr gut:

Femme de verre
Elyose
 •
Ipso Facto
 • 2015

----

implizit explizit, mit kraftvoller Musik:

Alexandria
Ignea
 •
The Sign of Faith
 • 2017

----

Kampf:

Planet War
Ignea
 •
Sputnik
 • 2013


Leviathan
Ignea
 •
The Sign of Faith
 • 2017

----

Wanderung ins Gefahrengebiet, Unheil droht:

Nomad's Luck
Ignea
 •
Nomad's Luck
 • 2023

----

Die Wölfe greifen an:

Jinnslammer
Ignea
 •
Jinnslammer
 • 2020

----

Finalszene: Fnbsrd starrt Vängä in die Augen und verhindert ihre Rache:

»Remember The Name«~– Fort Minor

---

Take Me Away
Ayron Jones
 •
Take Me Away
 • 2020

----

Kickstart My Heart
Motley Crue
 •
Dr. Feelgood
 • 1989

----

Good Luck
Broken Bells
 •
Good Luck
 • 2019

----

Selbstironie und Pazifismus, ein wenig Religionskritik:

Words Of War
Visions of Atlantis
 •
The Deep & The Dark
 • 2018

----

sehr ruhig mit Hoffnung:

Prayer To The Lost
Visions of Atlantis
 •
The Deep & The Dark
 • 2018

----

Endlich wieder etwas von Republica!


Out of the Darkness
Republica
 •
Republica
 • 1996

----

sehr schön:


Legends (feat. Chrigel Glanzmann)
Ad Infinitum
 •
Chapter III - Downfall
 • 2023

----

Frei wie der Wind – Santiano

----

Comeback aus Fnbsrd 1: "Fresenhof", diesmal das Santiano-Cover?

----

Der Text ist deutlich passender als erwartet:
"What I'm Made Of" – Crush 40

----

Bosskampf

2010 (feat. Dave Lombardo)
Apocalyptica
 •
7th Symphony
 •
2010

----

Fnbsrd 3 (die Reise über das Meer):

Legion Of The Seas
Visions Of Atlantis

Besonders schön durch Mischung aus weiblichem Gesang (curie) und männlichem Gesang.

(“Take the chart
Make a mark
…
Born to be free
We're the legion of the seas
We've conquered all the waters with our fleet”)

----

Melancholy Angel
Visions Of Atlantis

----

Fnbsrd’s eigene Titelmelodie:
Clocks
Visions Of Atlantis
(“I devise so many plans
How to rule my world, how to write my story
I enjoy all that I can
For happy times will turn to memories

I've been living in the past
Turned my lonely days into lonely future
Now, I leave it all behind
As the present time is the only one that lasts

Braving the clock takes its power away”)

----

Vängä und koroljow erzeugen einen Hurrikan im Mondlicht, das sich in Silberschuppen und Silberschwert bricht:
(“du setzt deine Windmagie bisher viel zu passiv ein”, befand Vängä)

Master The Hurricane
Visions Of Atlantis

----

Republica - Ready to Go Cover by Nera
NERA band
 • 19K views • 221 likes

https://music.youtube.com/watch?v=542lJUIpW_Y


Republica - Ready To Go (Cover by Frantic Romantic)
Frantic Romantic
 • 40K views • 380 likes


----

sehr hart:

Find You
CYGNOSIC
 •
Fire and Forget (Extended Edition)
 • 2013

----

Never Turn Back
Crush 40

What I'm Made Of…
Crush 40

----

Comeback aus dem Ende der IA2: “The Catalyst” von Linkin Park, ggf. als Halocene-Cover, das ist gut geworden (vielleicht kehrt gar eine der drei Protagonistinnen Ämändä Könstrüktä, Lätänä (?) Könstrüktä oder Rögü Kränk, von damals zurück. Oder Fnbsrd taucht kurz mitten in dieses Geschehen in einen Weltstein ab und befindet sich plötzlich im Reaktorraum nach dessen Zerstörung. Strahlung.)

----

New Divide
Linkin Park
und/oder Halocene-Cover

----

Holy Roller
Portugal The Man

----

weird, ytdl
Holy Roller (feat. Ryo Kinoshita)
Spiritbox
 •
Holy Roller
 •
2020

----

HELL YEAH ytdl

Dannie Histone - Atomic Man Cover (Portugal the man)
Dannie Histone
 •
625 views
 •
25 likes

----

(
ytdl Audio:

Aerosmith - Dream On (Vocal Cover)
Vicky Psarakis
)

(prüfen, dass Red Ivy’s Cover von “Modern Jesus” irgendwo in der Musikbibliothek liegt)

ytdl https://music.youtube.com/watch?v=Dk3boYl6qV4

ytdl https://music.youtube.com/watch?v=_trzqmAhTfs

----

Düsterer Instrumentalhimmel des Grauens:

2010 (feat. Dave Lombardo)
Apocalyptica
 •
7th Symphony
 •
2010

----

“not strong enough to stay away”

Not Strong Enough (feat. Brent Smith)
Apocalyptica
 •
7th Symphony
 •
2010

----

At The Gates Of Manala
Apocalyptica
 •
7th Symphony
 •
2010

----

Mary – The Happy Fits
fröhlich vorwärts in den Waldmorgen

----

Schön, grau mit Ultraviolatt:
"High Road" – Marnie
dringend mit Kopfhörern hören.

----

House of the Rising Sun –The Animals
Falls früher bereits genutzt, gerne als Comeback oder gar als Referenz auf die damalige Szene. <3

----

Sehr exotisch-schön:
"World Looking In" – Morcheeba

----

The Scene is Dead (Master Boot Record Remix) – Dubmood

----

Fnbsrd: Angriff auf den Nabenberg: "Uprising" – Muse

----

Rückkehr zur Hölle:

Going to Hell
– The Pretty Reckless

----

https://m.youtube.com/watch?v=EincW3c7dsc

Se tu m'ami - Giovanni Battista Pergolesi

----

ooooohu

Papercut
Halocene
 •
Papercut
 •
2023

#
Original ist von Linkin Park.

----

Summer of Luv (feat. Unknown Mortal Orchestra)
Portugal. The Man
 •
Chris Black Changed My Life
 •
2023

----

(dieses Lied und das ganze Album mal prüfen: )

Heavy Games II (feat. Jeff Bhasker)
Portugal. The Man
 •
Chris Black Changed My Life
 •
2023

----

Red Ivy - Modern Jesus (Portugal. The Man Cover)
Red Ivy

----

Fnbsrd:

See Who I Am
Within Temptation
 •
The Silent Force
 •
2004

----

Figure It Out
Royal Blood
 •
Royal Blood
 •
2014

----

// das Elfendorf stellt keine Krieger zur Verfügung, weil diese zur Verteidigung gegen den zu erwartenden Angriff benötigt werden. Vängäs Protest, dass bisher jede Verteidigung gescheitert und ein Angriff angezeigt sei, verhallt in der Stadthalle wie ein schlechtes Niesen im Wind.

Introkraft:

Shine a Little Light
The Black Keys
 •
"Let's Rock"
 •
2019

----

stark und schnell

Hot Blood
KALEO
 •
A/B
 •
2016

----

Mit Feuer.

Lo/Hi
The Black Keys
 •
"Let's Rock"
 •
2019

----

Shine
Camouflage
 •
Greyscale
 •
2015

----

Aha, endlich gefunden

Overload (Original Edit)
Sugababes
 •
Overload
 •
2000

----

Aber natürlich, das ist ja Fnbsrd in jeder Nacht:

Sleepwalking
The Birthday Massacre
 •
Pins And Needles
 •
2010

Und das Intro klingt so verdammt gut.

----

Science
The Birthday Massacre
 •
Walking with Strangers
 •
2007

----

Neu:

Call My Name
Beyond The Black
 •
Call My Name
 •
2023

----

Wish I Had an Angel
Nightwish
 •
Wish I Had An Angel (EU Version)
 •
2004

----

Burning Up
Ladytron
Velocifero
2008

----

DAS Cover des Linkin-Park-Songs, der beim Abflug von Eta Carinae ABC X läuft. DAS muss rein. Weiblicher Gesang in dem Lied ist absolut genial. // “\\In Erinnerung an Chester Bennington” in die Musikliste schreiben (Extrazeile)

Castle of Glass
Halocene
 •
Castle of Glass
 •
2022

Und auf jeden Fall auch:

Breaking the Habit
Halocene
 &
First to Eleven
 •
Breaking the Habit
 •
2022

Und:

Bring Me To Life
Halocene
 •
Bring Me To Life: Evanescence Tribute
 •
2022

Und:

Call Me When You're Sober
Halocene
 &
Violet Orlandi
 •
Bring Me To Life: Evanescence Tribute
 •
2022

Auch sehr schön:

Toss a Coin to Your Witcher
Cover by Barbie Sailers and Violet Orlandi
 •
Toss a Coin to Your Witcher
 •
2020

----

Latein im Refrain:

The Mind Maelstrom
Sirenia
 •
The 13th Floor
 •
2009

----

Vängä erlegt bei einer nächtlichen… “Jagd”… den Sanitäter, der bewusst dafür einen Nachtspaziergang durch den Wald unternimmt:

Maneater
Daryl Hall & John Oates
 •
H2O
 •
1982

----

yt-dlp mit Video:
https://music.youtube.com/watch?v=-rKOoM7S6mw

----

Absolute Power. Die Melodie am Anfang läuft auch als Soundtrack in meinem uralten Computerspiel “Robots” / “Robots 2” / “Robots 3”, glaube ich. Wenn ich bloß wüsste, wie das Original heißt!

In Control
Nemesea
 •
In Control
 •
2007

----

Die Melodie am Anfang ist offenbar inspiriert durch, oder ein Cover von, "Africa" von Toto:

Afterlife
Nemesea
 •
The Quiet Resistance
 •
2011

----

Sehr speziell: Vängä steht auf den Baumkronen und überblickt den Wald auf der Suche nach einem silbernen Drachen. (“I don’t know where to go”)

Caught in the Middle
Nemesea
 •
The Quiet Resistance
 •
2011

----

Mmmmmmmmmh:

Morgenluft am Waldsee, Reflektion von Infrarot im Spiegel der Seele, Akzente aus Silber:

Sun is up
INNA
 •
Sun is up
 •
2022

----

Mir müsste nochmal jemand erklären, warum Fnbsrd 2 noch keinen Oasis-Song enthält. Eh?

Wie wäre es mit…

D'You Know What I Mean? (NG's 2016 Rethink)
Oasis
 •
Be Here Now (Deluxe Remastered Edition)
 •
1997

Und dazu dann, das ist genial: Fnbsrd stößt zu Floating Island an die Stelle, an der er auf den verlassenen Gleisen liegt. Der Baum ist verschwunden, der Zug auch… Aus dem Wald beobachtet Fnbsrd den Protagonisten/Antagonisten der IA2/3, wie er ins Gebirge zieht. Ins Gebirge! Elfenwald…

Floating Island ist der Einsiedler im Gebirge. Science-Fiction- und Fantasywelt verschmelzen. Ab und zu gibt es Risse im Film, schlecht geflickte Stellen, an denen die Kunstwerke vereint wurden.

----

Intro hart und melodisch: curie wartet in Last Hope auf koroljow, in grünen Schuppen:

Far Away
Elysion
 •
Bring Out Your Dead
 •
2023
Liedtext:
Trapped in a skin that doesn't fit, hear me
…
Lies the dream you've set your eyes on
Lies the paradise your heart wants
If you sail far away
…
So far away

----

Düster!
Blink Of An Eye
Elysion
 •
Bring Out Your Dead
 •
2023

----

Crossing Over
Elysion
 •
Crossing Over
 •
2023

—--------

Killing My Dreams
Elysion
 •
Silent Scream
 •
2009

—--------------

Oh, bitte.

Never Forever
Elysion
 •
Silent Scream
 •
2009

-----

Absent Without Leave
Sirenia
 •
Nine Destinies and a Downfall
 •
2007

Liedtext:
I can feel the eyes of your despair
Even where you're not here
…
There's a shadow that dwells inside your head
Within mine as well
Turns our lives into a living hell
Like a curse and a spell
I've been drifting away from days of light
I've been swept away far into the night
I guess I'll never ever make it back
…

----

Fnbsrd-schwarz:

Dominion / Mother Russia (New Version for Digital)
The Sisters Of Mercy • Floodland • 1985

----

Within Temptation -Skyfall (Adele Cover)

----

She Is My Sin
Nightwish
 •
Wishmaster
 • 2000

 ----

warum noch nicht drin?

Never Forever
Elysion
 •
Silent Scream
 • 2009

----

Schräg-hart-schön:

The Last Call
Sirenia
 •
Nine Destinies and a Downfall
 • 2007

----

Neue Titelmelodie (!):

Hand of Sorrow
Within Temptation
 •
The Heart Of Everything
 • 2007

(Text und Symphonie gegen 1:00 - Ende, Klavier und Chor/Geigen)

Neues Fnbsrd-Theme (!):

See Who I Am
Within Temptation
 •
The Silent Force
 • 2004

Dazu “Lost” auch von Within Temptation

----

Around The World
MOUNT
 &
Noize Generation
 •
Around The World
 •
2018


(Cover von ATC’s “Around The World”)

----

Aaron Smith - Dancin - Krono Remix ft. Luvli

----

Vermutlich bereits Teil der Liste:

New Evil
Nocturna
 •
New Evil
 •
2021

----

Hart (selbst-)zerstörerischer Song: Möglicherweise Titelmelodie des Angreifers mit den Wölfen.

Ghost Myself
DIAMANTE
 •
Ghost Myself
 •
2020

----

Interessanter Metalsong:

Bye Bye Bye
Anette Olzon
 •
Strong
 •
2021

----

Believe (feat. Goodboys)
ACRAZE
•
Believe
• 2022

----

irgendein unbekannter Nirvana-Song.

----

(woher kommen die ge-remixten Chorstimmen?)

Lay Low
Tiësto
 •
DRIVE
 •
2023

----

The Path to Decay

----

Burning Up
Ladytron
 •
Velocifero
 •
2008

----

Dancin (Krono Remix) (feat. Luvli)
Aaron Smith
 •
Dancin (Krono Remix)
 •
2005

----

Trauer und verbittertes Schlagzeug gegen Mitte:

(Zerstörungsanblick)

Fire and Ice
Within Temptation
 •
The Unforgiving
 •
2011

----

“Lost” von Within Temptation ist viel schöner als in Erinnerung

----

Unterwasser-Abenteuer (wow, tief):
Deep Blue
Ladytron
 •
Velocifero
 •
2008

----

musikalisch bunt

Disenchantment
Ignea
 •
Disenchantment
 •
2020

----

Ups, das hier ist das Original des vermeintlichen Sirenia-Songs “What Else Is There?”

What Else Is There ?
Röyksopp
 •
The Understanding
 •
2005

Also natürlich das Original verwenden!

----

in der Mitte sehr schönes Gitarrenspiel

No End, No Beginning
Poets of the Fall
 •
Alchemy, Vol. 1
 •
2011

----

Telephone (feat. Beyoncé)
Lady Gaga
•
The Fame Monster (International Deluxe)
• 2009

----

oooooh
ookay
Can't Fight The Moonlight
LeAnn Rimes
•
Coyote Ugly
• 2000

----

möglicherweise… vielleicht. Für Vängä.
Scandalous [StarGate Radio Mix] (feat. StarGate)
Mis-Teeq
 •
Eye Candy [Deluxe]
 •
2003

----
“Toxic” - Britney Spears
----

Titelmelodie des zweiten Teils von Fnbsrd 2: “Stronger” von den Sugababes, wegen der Atmosphäre und den Geigen, und natürlich wegen dem Text

----


Ach, "Katyusha" heißt dieses Volkslised.
Katjuscha (russisch Катюша Katharinchen) ist ein russisches Liebeslied.

Katyusha (Extended Mix)
Paradoqz
 •
Katyusha (Extended Mix)
 • 2020

—

weird Chortechno

Voices
T78
 &
Mariana BO
 •
Voices
 • 2024

——


"slowed"?!

ICONIC (SLOWED)
Crème
 • 
1.2K views
 • 
49 likes


—

lol wie bescheuert

Blonde Chaya (Sped Up)
Gringo Bamba
 & 
Amaru
 • 
1.5M views
 • 
22K likes


—


PULS - Kämpferherz

PULS - Alles was bleibt

—

Warum um alles in der Welt kannte ich "It's the Fear" von Within Temptation noch nicht?

Wow.

—

Ui!

das Intro ist offenbar ein Zitat von

Time Back – Bad Style

und auch "Summertime Sadness" (I feel it in the air) von Lana Del Rey ist drin.

hier:

Nightcore – Sturm der Fantasie (Lyrics + Translation)

Sturm der Fantasie
Franziska Wiese
 • 
Sinfonie der Träume
 • 
2016


—

<3 hoffnungsvolles, frohes Sehnsuchts-/Liebeslied

Westwind - DArtagnan

—

als Nightcore entdeckt:

Celldweller – Breakout

—

Was du liebst (Orchesterversion)
Santiano
 • 
Die Sehnsucht ist mein Steuermann - Das Beste aus 10 Jahren
 • 
2022

—

Iris - Annie, Would I Lie To You?

 • 
157K views
 • 
1.7K likes


—

Funky Cheesy Fnbsrd Theme:

Science
Ashbury Heights
 • 
The Victorian Wallflowers
 • 
2018



—

Nothing But A Fool
New Order
 • 
Music Complete
 • 
2015



—

Genie In a Bottle
Christina Aguilera
 • 
Christina Aguilera (Expanded Edition)
 • 
2000



—


weird

Cha Cha Cha
Käärijä
 • 
Cha Cha Cha
 • 
2023


—

Ready Or Not (Radio Edit)
Cascada
 • 
Evacuate The Dancefloor
 • 
2011



—

Happy Nation
Ace of Base
 • 
Happy Nation
 • 
1992




—

Alpenglow
Nightwish
 • 
Endless Forms Most Beautiful
 • 
2015


—

Turn Loose The Mermaids
von Nightwish

auch als Nightcore wunderschön. Nicht schnell. Aber wunderschön.

Der Mittelteil klingt nach Ennio Morricones besten Liedern.

—

Wunderschönes Intro, abtauchend!

Mirror of Night
Delain
 • 
Dark Waters
 • 
2023



—

Schönes Cover, ytdl

Radioactive
Within Temptation
 • 
Hydra (Deluxe Edition)
 • 
2014



—
Ouh. Hübscher Remix. Scooter hat das auch einmal geremixt, wie heißt das Original überhaupt?

Chase The Sun
Koven
 • 
Chase The Sun
 • 
2023


—

<3 Flöte oder sowas

I Want My Tears Back
Nightwish
 • 
Imaginaerum
 • 
2011


—

2:20 plötzlich oha
Escapist
Nightwish
 • 
Bye Bye Beautiful
 • 
2008


—

Monsters
Visions of Atlantis
 • 
Monsters
 • 
2024


—

Better Not Lie To Me
Garbage
 • 
Lie To Me
 • 
2024

—

Queen of Shadow
Delain
2023

—

Elektrisches Gefühl
Juli
 • 
In Love
 • 
2010

—

Bar-/Kneipensong:

Skal
Miracle of Sound

—

mystisch-mysteriös:

Das Würfelspiel
Juliane Werding

—

Celestial Sphere
Atom Music Audio
 •
EON II
 • 2019

—

Spuren auf dem Mond <3

—

Wild und frei (die Hexe Runa)
Corvus Corax

Wild und frei
was ist schon dabei
Hexenleben ganzes Streben
lol
Vängä?

—

Weißt du, wer ich bin, weißt du wer ich war, Bettler oder Königin

Weisst du, wer ich bin (Neu)
Juliane Werding
 •
Land Der Langsamen Zeit
 • 1997

—

Pirate Queen
(von wem?)

—

Turn Loose The Mermaids
Nightwish

----

The Scent Of Dead Leaves
Rage Of Light
•
The Scent Of Dead Leaves
• 2023

----

Fnbsrd-2-Theme (abspielen beim Auszug aus Last Hope zum Elfenwald):

Is there a hero somewhere, someone who appears and saves the day?
Someone who holds out a hand and turns back time?
Is there a hero somewhere, someone who will never walk away?
Who doesn't turn a blind eye to a crime?

Locking Up the Sun
Poets of the Fall
 •
Carnival of Rust
 •
2006

----

...
United in silent resistance
Of bowing to false kings
So let me run to your shelter tonight
Run from this meaningless pantomime
I'll swallow my pride
Give up the pretence of bowing to false kings

Bought their smiles, liquid and smooth
Took their words for the truth
Edge of light and shade
My broken soul once more enslaved

False Kings
Poets of the Fall
 •
Ultraviolet
 •
2018

----

vermutlich bereits in der Liste:
Denial
Sugababes
 •
Change
 •
2007

Aber dazu Kontext: Vängä vermisst Fnbsrd… Der Elf ist nur ein halber Ersatz.

----

Neue Brücken
PUR
 •
Seiltänzertraum
 • 1993

Bedingung für eine Hochzeit mit mir wäre, dass dieses Lied in einer Kirche gesungen wird. Dafür allein würde ich »Ja« sagen.

----

</~ Musik-Notizzettel ~>


\part{Bonusmaterial}

\chapter {Postskriptum}

Noch nicht geschrieben.


\chapter{Titelmelodie}

\begin{figure}[p]
    \includegraphics[width=\textwidth, page=1]{z-include-fnbsrd2theme.pdf}
\end{figure}


\chapter{Musikliste}

\textbf{Für Filmproduzenten, Träumer und Multitasking-Genies.}

\begin{itemize}
    \item Falls Du ernsthaft einen Film zu diesem Buch drehen möchtest.
    \item Falls Du das gesamte Buch bereits ausgelesen hast und die genannten Lieder vielleicht noch nicht kennst. Höre die Lieder und stelle Dir dabei die Szenen vor. Wenn es schon keinen Fnbsrd-2-Film gibt, kannst Du wenigstens einen Film in deinem Kopf laufen lassen.
    \item Falls Du beim Lesen Musik hören möchtest, die zur aktuellen Szene passt.
\end{itemize}

Diese Liste wurde von Tobias Frei zusammengestellt und impliziert keinerlei Unterstützung oder Befürwortung durch die Komponisten der Lieder. Eines Tages wird jedes dieser Lieder in die Gemeinfreiheit übergehen; der genaue Zeitpunkt hängt von verschiedenen Gesetzen ab.

\begin{enumerate}
    \item Titelmelodie:\\ »Blizzard«~– Tobias »ToBeFree« Frei
    \item \textbf{Teil 1: Numeri.}\\ »Ribbons«~– The Sisters of Mercy
    \item Aufbruch zum Elfenwald:\\ »The Cold«~– Delain
    \item Kartenlesen im Schnee:\\ »Our Destiny«~– Epica
	\item Willkommen in Majoris:\\ »Temple of Love«~– The Sisters of Mercy
	\item Tempel der Liebe:\\ »How Soon is Now?«~– The Smiths
	\item Aquarii am Waldesrand:\\ »A Forest«~– The Cure
	\item Dunkelheit und Wölfe:\\ »Trouble’s Coming«~– Royal Blood
	\item MAGNUS’ Vortritt:\\ »Legends Never Die«~– Against The Current
    \item Kömbüsö:\\ »Strangelove«~– Depeche Mode
    \item Reflexion:\\ »Don’t Look Back in Anger«~– Oasis
	\item Zu siebt in den Wahnsinn: »Ground«~– Assemblage 23
    \item Dunkeldiskussion:\\ »See Me in Shadow«~– Delain
    \item Weiter des Wegs:\\ »I Was Wrong«~– The Sisters of Mercy
    \item 3 AM:\\ »The Noise Inside My Head«~– Assemblage 23
    \item Der Gang zum Berggipfel:\\ »Moonchild«~– Fields of the Nephilim
    \item Drachenflug zu zweit:\\ »Never Forever«~– Elysion
    \item Ich kann nie wieder fliegen:\\ »Fliegen«~– Blutengel
    \item \textbf{Teil 2: Deuteronomium.}\\ »Lost in Forever«~– Beyond the Black
    \item In alle Ewigkeit:\\ »In Alle Ewigkeit«~– Blutengel
    \item Sanfter Regen unter Nadelbäumen:\\ »A Little Bit Off«~– Five Finger Death Punch
    \item Verfolgungsjagd durch den Wald, Teil 1:\\ »...«~– XXXX
    \item Weltfall:\\ »...«~– XXXX
    \item Verfolgungsjagd durch den Wald, Teil 2:\\ »Out In The Fields«~– Gary Moore und Phil Lynott
    \item Zweihundert Wölfe:\\ »Ritual«~– In This Moment
    \item Konfrontation:\\ »In the Air Tonight«~– In This Moment (Original: Phil Collins)
    \item Violettes Schwarz:\\ »The In-Between«~– In This Moment
    \item Erwachen zwischen Hochhäusern:\\ »Die letzte Fahrt«~– Santiano\\Das Lied ist wunderschön, auch wenn mir das Weltall näher als Wellen ist. Ich möchte, dass es eines Tages zum Abschied gespielt wird.

    \item Am Eisboden:\\ »Lost in Life«~– Sirenia

    \item Ende der Zeitebene:\\ »Black Wedding« (»White-Wedding«-Parodie)~– In This Moment mit Rob Halford
    \item Realitätsbruch zwischen Hochhäusern:\\ »High«~– Sir Sly
    \item Cliffhanger zwischen Wolkenkratzern:\\ »Kein Zurück«~– Peter Heppner und Markus Reinhardt


    \item Philosophie am Lagerfeuer:\\ »Alles«~– Blutengel

    \item Handelsüblicher Ostbahnhof:\\ »Das Würfelspiel«~– Juliane Werding

    \item Zeitreisendenparty in Zimmer 205:\\ »Hart am Ball (Live)« (vom Album »100\% Leben«)~– Achim Reichel
    \item Rosa Luftballon:\\ »Einmal nur mit Erika«~– Hubert Kah
    \item Feuer gegen Schnee:\\ »Hand in Hand«~– Unheilig
    \item Silberflug:\\ »Eternal Rains«~– Ad Infinitum
    \item Umzingeltes Duo:\\ »Bosorkun«~– Ignea
    \item Technische Eskalation:\\ »Mad Matrix« (aus »Shadow The Hedgehog«)~– Jun Senoue

    \item Weiter zum Gipfel:\\ »Wir sind die Gipfelstürmer«~– Unheilig
	\item Ruinen im Schnee:\\ »Gollum’s Song«~– Emiliana Torrini
    \item  :\\»Fires of Hell (Remains of Blazing Hell)« aus Touhou: Scarlet Curiosity~–Hachimitsu Lemon, basierend auf den Werken von Jun’ya Ota (ZUN)
    \item  :\\»Laissez-faire Youkai Castle (Palace of Earth Spirits)« aus Touhou: Scarlet Curiosity~–Hachimitsu Lemon, basierend auf den Werken von Jun’ya Ota (ZUN)

    \item Erinnerung an curie:\\ »Kämpferherz«~– Staubkind und Iris Mareike Steen
    \item ...:\\ »...«~– ...

    Der Weg erweist sich als beschwerlich und mündet in eine Lawine. Eine wenig stabil wirkende Höhle mit antiken Stufen ist der einzige Ausweg durch den Bergkamm. Immerhin hat sie die Lawine einigermaßen überstanden.

    \item Folge dem weißen Kaninchen:\\ ?
    \item Hoffnung:\\ »Endless«~– The Birthday Massacre

    \item Letzter Schleichweg zum Berggipfel, Sonnenuntergang, Showdown-Beginn:\\ »Téoura«~– Ignea
    \item Die Hütte des Bergkönigs:\\ »Planet War«~– Ignea
    \item Dunkelheit:\\ »Sweet Dreams«~– Aviators
    \item Fnbsrd schützt den Bergkönig vor Vängä:\\ »Remember The Name«~– Fort Minor



    \item Rückblick auf verhängnisvolle Fehler:\\ »I Was Wrong«~– The Sisters of Mercy
    Der Bergkönig, der böse Elf, das Ziel der Mission, der vermeintliche Endboss... muss befreit werden. Gerettet aus den Fängen seiner eigenen Kreaturen, denn die Wölfe haben sich durch ihr stumpfes Handeln gegen ihn gewandt: Er sitzt gewissermaßen im goldenen Käfig, bekommt blutiges Gold geliefert und kann das Areal nicht verlassen, da alles außerhalb des Areals zum Jagdgebiet der Wölfe gehört und er dort sofort angegriffen würde. Man sieht ihm an, dass er mindestens einmal vergeblich versucht hat, zu fliehen. Er benötigt langfristige Hilfe im nächsten noch lebenden Dorf.
    »Herr, die Not ist groß!
    Die ich rief, die Geister
    werd ich nun nicht los.« (Goethe: Der Zauberlehrling)
    \item Euphorie der Freiheit:\\ »Extreme Ways (Jason Bourne)«~– Moby
    \item Krankentransport ins Dorf:\\ »Liebe gewinnt«~– Brings und das Beethoven Orchester Bonn
    \item Sonnenaufgang im Waldgebirge:\\ Firefly (Hotaru, 蛍)~– Maiko Fujita (藤田麻衣子)
    \item Outro:\\ »Sing«~– Blutengel

\end{enumerate}



“Varyags Of Miklagaard” von Amon Amarth

“Back to Black” von Amy Winehouse

“Back in Black” von AC/DC

“Hell’s Bells” von AC/DC

Irgendwas von Airbourne.

“Move It” von Jon Void und Coastal

“Sing” von Blutengel.

“Fire in the Distance” von Blutengel: Titelsong für Fnbsrd 2 oder einen der Buchteile?

“Move It” von Jon Void und Coastal

Stark inspirierend für koroljow und curie: “Fliegen” von Blutengel. Der Text und die Struktur des erst abwechselnden und dann gemeinsamen Gesangs.

“Deine Welt” von Blutengel

“Maleficent” von Ad Infinitum

“New Evil” von Nocturna ist wirklich gut geworden. Musikvideo herunterladen.
“World Coming Down” von Ashbury Heights
“Storm the Sorrow” von Epica
“New Frontier” von Miracle of Sound und Karliene
Fnbsrd 3 Boss: “Breathe” von Prodigy. Text passt gut zu Bosskampf.
Fnbsrd gegen einen merkwürdigen Seher/Propheten/Heuchler: “John the Revelator” von Depeche Mode!
“Drive” von Assemblage 23
“Far Far Away” von Battle Beast (Cover?)
“Sancta Terra” von Epica
“Faint” von Linkin Park (der Beginn wurde doch von Icon for Hire gecovert… wo…)
Curie und Silberdrache: “World on Fire” von Battle Beast
“Crosstalk” von Assemblage 23.
“F-Zero: Mute City”
“Marching on Versailles” von Ad Infinitum




SoulOfTheInternet und Fnbsrd surfen \emph{auf} einer Straßenbahn in der Zukunftsstadt, Stromschiene ist am Boden zwischen den Gleisen. Verfolgungsjagd quer durch die Stadt. "Sofort anhalten oder wir schießen". Darauf hielten sie natürlich nicht an. (IA1-Parodie) »The Flute Tune: Soulpride Remix«~– Jaycut und Kolt Siewerts


Musik, die leider nicht mehr in die Musikliste gepasst hat:

\begin{itemize}
    \item :\\ »Everstrong«~– Eye of Melian
    \item :\\ »More«~– The Sisters of Mercy
    \item :\\ »Doctor Jeep«~– The Sisters of Mercy
    \item :\\ »The Golden Shell«~– Ignea
    \item :\\ »The Great Commandment«~– Camouflage
    \item :\\ »Underland«~– Delain
    \item :\\ »Ravenous«~– Ad Infinitum
    \item :\\ »Ruf der Freiheit«~– dArtagnan
    \item :\\ »Natural Blues«~– Moby
    \item :\\ »Lift Me Up«~– Moby
    \item :\\ »Breaking the Habit«~– Linkin Park
    \item :\\ »Wake the White Wolf«~– Miracle of Sound
    \item :\\ »Shadow of the Ash (Hard Vocal)«~– Miracle of Sound
    \item :\\ »The Tale of Cu Chulainn«~– Miracle of Sound
    \item :\\ »This Ain’t Good«~– She Hates Emotions
    \item :\\ »Legends«~– Ad Infinitum und Chrigel Glanzmann
    \item :\\ »Phenomenon«~– Otyken
    \item :\\ »Mad World«~– Within Temptation
    \item :\\ »Mercy Mirror«~– Within Temptation
    \item :\\ »Minne«~– Oonagh und Santiano
    \item :\\ »Ignorance«~– Paramore
    \item :\\ »Faint«~– Linkin Park
    \item :\\ »Ready Steady Go«~– Paul Oakenfold
    \item :\\ »The One I Love«~– She Hates Emotions
    \item :\\ »Fairytale of Doom«~– Beyond the Black
    \item :\\ »Noise Pollution«~– Portugal. The Man, Mary Elizabeth Winstead und Zoe Manville
    \item :\\ »Let the Wind Erase Me«~– Assemblage 23
    \item :\\ »Horizon«~– Assemblage 23
    \item :\\ »Architect of Paradise«~– Ad Infinitum
    \item :\\ »Songs of Love and Death«~– Beyond the Black
    \item :\\ »Open Your Eyes«~– Guano Apes
    \item :\\ »Lift«~– Poets of the Fall
    \item :\\ »Two Worlds«~– Xandria
    \item :\\ »Paid in Full«~– Sonata Arctica
    \item :\\ »Fighting in Built-Up Areas«~– Ladytron
    \item :\\ »Dernière danse«~– Indila
    \item :\\ »Red Stars«~– The Birthday Massacre
    \item :\\ »Goodnight«~– The Birthday Massacre
    \item :\\ »Skyfall«~– Adele
    \item :\\ »Rockafeller Skank«~– Fatboy Slim
    \item :\\ »Like Fear, Like Love«~– The Birthday Massacre
    \item :\\ »Drowning in Darkness«~– Beyond the Black (extrem gut vor allem am Anfang und durch den Titel)
    \item :\\ »The Other Side of the Wall«~– Assemblage 23
    \item :\\ »Big Bad Wolf«~– In This Moment
    \item :\\ »Mechanical Instinct«~– Aviators
    \item :\\ »City of Angels«~– Ladytron (neues 2023er Album auf gute neue Songs überprüfen)
    \item :\\ »In jener Nacht«~– dArtagnan
    \item :\\ »Flucht nach vorn«~– dArtagnan
    \item :\\ »Korobeiniki«~– dArtagnan (wirklich gelungen)
    \item :\\ »Headlights«~– Ashbury Heights
    \item :\\ »I Miss the Misery«~– Halestorm
    \item :\\ »Our Solemn Hour«~– Within Temptation
    \item :\\ »Self-Fulfilling Prophecy«~– Solitary Experiments
    \item :\\ »Wonderland«~– Solitary Experiments
    \item :\\ »Call the Ships to Port«~– Covenant
    \item :\\ »The Mountain«~– Ladytron
    \item :\\ »Synchronize (SynchroMeshMix)«~– De/Vision
    \item :\\ »Through the Fire and Flames«~– DragonForce
    \item :\\ »The Last Dragonborn«~– DragonForce
    \item :\\ »Voulez-Vous«~– ABBA
    \item :\\ »Informer«~– Snow
    \item :\\ »Ascension«~– Miracle of Sound und Karliene
    \item :\\ »A Prophecy of Worlds to Fall«~– Xandria (dunkler Chor, Weltuntergang)
    \item :\\ »Out of Control«~– Battle Beast (schönes Gitarrensolo bei 2:30)
    \item :\\ »In Black and White«~– Sonata Arctica
    \item :\\ »Move On Baby«~– Capella
    \item :\\ »Keep On«~– Portugal. The Man
    \item :\\ »Magura’s Last Kiss«~– Ignea
    \item :\\ »Master of Illusion«~– Battle Beast
    \item :\\ »Nothing Left«~– Delain und Marco Hietala
    \item :\\ »Hardcore Vibes (Extended Mix)«~– Ravers on Dope
    \item :\\ »Blinded by Hatred«~– Unsun
    \item :\\ »Face the Truth«~– Unsun
    \item :\\ »The Haunting (Somewhere in Time) (feat. Simone Simons)«~– Kamelot
    \item :\\ »Friend or Foe?«~– Riverside
    \item :\\ »Don’t Say a Word (Sonata Arctica Cover)«~– Xandria
    \item :\\ »Hallelujah«~– Beyond The Black
    \item :\\ »Storytime«~– Nightwish
    \item :\\ »Lost«~– Linkin Park
    \item :\\ »Untraveled Road«~– Thousand Foot Krutch
    \item :\\ »Heart of the Hurricane«~– Beyond the Black
    \item :\\ »Through the Mirror«~– Beyond the Black (eher traurig, Fnbsrd-Theme)
    \item :\\ »Familiar Hell«~– Battle Beast
    \item :\\ »Federkleid«~– Faun
    \item :\\ »Viva la Vida«~– Gregorian
    \item :\\ »Rainbow in the Sky«~– Paul Elstak
    \item :\\ »13 Tage«~– Klaus-Peter Schweizer
    \item :\\ »Tattoo«~– Loreen
    \item :\\ »«~– 
    \item :\\ »Wide Awake«~– Beyond the Black (schöne Ballade)
    \item :\\ »Heart of the Hurricane«~– Beyond the Black
    \item :\\ »Alexandria«~– Ignea
    \item :\\ »My Down-to-Earth Lover«~– Blackbriar
    \item :\\ »Everytime We Touch (Hardwell & Maurice West Remix)«~– Cascada
    \item :\\ »Around the World«~– A Touch of Class
    \item :\\ »So still mein Herz«~– Oonagh
    \item :\\ »Castle of Glass«~– Linkin Park
    \item :\\ »Immortal«~– Blutengel
    \item :\\ »Alle Wunden«~– Blutengel
    \item :\\ »Solitude«~– Dubmood
    \item :\\ »Freestyler«~– Boomfunk MC’s
\end{itemize}


\chapter{Bildquellen}

Alle verwendeten Bilder sind gemeinfrei. Die Verwendung der Bilder in diesem Roman impliziert keinerlei Unterstützung oder Befürwortung durch ihre Schöpfer.

\begin{itemize}
    \item \textbf{Buchcover:} CC0-Lizenz / Public Domain.\\ Xyzxyz Xzyzxy (XXXXXXXXX, Pixabay vor 2019) % TODO
    \item \textbf{Titelbild des zweiten Teils:} CC0-Lizenz / Public Domain.\\ Xyzxyz Xzyzxy (XXXXXXXXX, Pixabay vor 2019) % TODO
\end{itemize}


\chapter{Lizenz des Buchinhalts}

\textbf{Fnbsrd 2 © by\\ Tobias Frei, fnbsrd.de}

Dies ist eine offizielle Ausgabe des Romans Fnbsrd, herausgegeben von Tobias Frei. Veränderte Versionen und unautorisierte Nachdrucke müssen deutlich als solche erkennbar sein. Auch das Impressum muss angepasst werden, wenn das Dokument verändert wird.

Falls du die Rechte in dieser Lizenz nutzen möchtest, musst du sie vollständig gelesen und verstanden haben. Es genügt nicht, nur eine Zusammenfassung zu lesen. Aus diesem Grund wird in diesem Buch keine Zusammenfassung angeboten.

This novel is licensed under a Creative Commons Attribution-ShareAlike 4.0 International License.

You should have received a copy of the license along with this work. If not, see\\
https://creativecommons.org/licenses/by-sa/4.0/legalcode

You are required to actually read and understand the full text of the license, not a summary.

\begin{center}
    \large{\textbf{Creative Commons Attribution-ShareAlike 4.0 International Public License}}
\end{center}

By exercising the Licensed Rights (defined below), You accept and agree to be bound by the terms and conditions of this Creative Commons Attribution-ShareAlike 4.0 International Public License ("Public License"). To the extent this Public License may be interpreted as a contract, You are granted the Licensed Rights in consideration of Your acceptance of these terms and conditions, and the Licensor grants You such rights in consideration of benefits the Licensor receives from making the Licensed Material available under these terms and conditions.

\begin{center}
    \textbf{Section 1 -- Definitions.}
\end{center}

\begin{itemize}
    \item[a.] \textbf{Adapted Material} means material subject to Copyright and Similar Rights that is derived from or based upon the Licensed Material and in which the Licensed Material is translated, altered, arranged, transformed, or otherwise modified in a manner requiring permission under the Copyright and Similar Rights held by the Licensor. For purposes of this Public License, where the Licensed Material is a musical work, performance, or sound recording, Adapted Material is always produced where the Licensed Material is synched in timed relation with a moving image.
    \item[b.] \textbf{Adapter's License} means the license You apply to Your Copyright and Similar Rights in Your contributions to Adapted Material in accordance with the terms and conditions of this Public License.
    \item[c.] \textbf{BY-SA Compatible License} means a license listed at creativecommons.org/compatiblelicenses, approved by Creative Commons as essentially the equivalent of this Public License.
    \item[d.] \textbf{Copyright and Similar Rights} means copyright and/or similar rights closely related to copyright including, without limitation, performance, broadcast, sound recording, and Sui Generis Database Rights, without regard to how the rights are labeled or categorized. For purposes of this Public License, the rights specified in Section 2(b)(1)-(2) are not Copyright and Similar Rights.
    \item[e.] \textbf{Effective Technological Measures} means those measures that, in the absence of proper authority, may not be circumvented under laws fulfilling obligations under Article 11 of the WIPO Copyright Treaty adopted on December 20, 1996, and/or similar international agreements.
    \item[f.] \textbf{Exceptions and Limitations} means fair use, fair dealing, and/or any other exception or limitation to Copyright and Similar Rights that applies to Your use of the Licensed Material.
    \item[g.] \textbf{License Elements} means the license attributes listed in the name of a Creative Commons Public License. The License Elements of this Public License are Attribution and ShareAlike.
    \item[h.] \textbf{Licensed Material} means the artistic or literary work, database, or other material to which the Licensor applied this Public License.
    \item[i.] \textbf{Licensed Rights} means the rights granted to You subject to the terms and conditions of this Public License, which are limited to all Copyright and Similar Rights that apply to Your use of the Licensed Material and that the Licensor has authority to license.
    \item[j.] \textbf{Licensor} means the individual(s) or entity(ies) granting rights under this Public License.
    \item[k.] \textbf{Share} means to provide material to the public by any means or process that requires permission under the Licensed Rights, such as reproduction, public display, public performance, distribution, dissemination, communication, or importation, and to make material available to the public including in ways that members of the public may access the material from a place and at a time individually chosen by them.
    \item[l.] \textbf{Sui Generis Database Rights} means rights other than copyright resulting from Directive 96/9/EC of the European Parliament and of the Council of 11 March 1996 on the legal protection of databases, as amended and/or succeeded, as well as other essentially equivalent rights anywhere in the world.
    \item[m.] \textbf{You} means the individual or entity exercising the Licensed Rights under this Public License. Your has a corresponding meaning.
\end{itemize}

\begin{center}
    \textbf{Section 2 -- Scope.}
\end{center}

\begin{itemize}
    \item[a.] \textbf{License grant.}
    \begin{itemize}
        \item[1.] Subject to the terms and conditions of this Public License, the Licensor hereby grants You a worldwide, royalty-free, non-sublicensable, non-exclusive, irrevocable license to exercise the Licensed Rights in the Licensed Material to:
        \begin{itemize}
            \item[A.] reproduce and Share the Licensed Material, in whole or in part; and
            \item[B.] produce, reproduce, and Share Adapted Material.
        \end{itemize}
        \item[2.] \underline{Exceptions and Limitations}. For the avoidance of doubt, where Exceptions and Limitations apply to Your use, this Public License does not apply, and You do not need to comply with its terms and conditions.
        \item[3.] \underline{Term}. The term of this Public License is specified in Section 6(a).
        \item[4.] \underline{Media and formats; technical modifications allowed}. The Licensor authorizes You to exercise the Licensed Rights in all media and formats whether now known or hereafter created, and to make technical modifications necessary to do so. The Licensor waives and/or agrees not to assert any right or authority to forbid You from making technical modifications necessary to exercise the Licensed Rights, including technical modifications necessary to circumvent Effective Technological Measures. For purposes of this Public License, simply making modifications authorized by this Section 2(a)(4) never produces Adapted Material.
        \item[5.] \underline{Downstream recipients}.
        \begin{itshape}\begin{itemize}
            \item[A.] \underline{Offer from the Licensor -- Licensed Material}. Every recipient of the Licensed Material automatically receives an offer from the Licensor to exercise the Licensed Rights under the terms and conditions of this Public License.
            \item[B.] \underline{Additional offer from the Licensor -- Adapted Material}. Every recipient of Adapted Material from You automatically receives an offer from the Licensor to exercise the Licensed Rights in the Adapted Material under the conditions of the Adapter's License You apply.
            \item[C.] \underline{No downstream restrictions}. You may not offer or impose any additional or different terms or conditions on, or apply any Effective Technological Measures to, the Licensed Material if doing so restricts exercise of the Licensed Rights by any recipient of the Licensed Material.
        \end{itemize}\end{itshape}
        \item[6.] \underline{No endorsement}. Nothing in this Public License constitutes or may be construed as permission to assert or imply that You are, or that Your use of the Licensed Material is, connected with, or sponsored, endorsed, or granted official status by, the Licensor or others designated to receive attribution as provided in Section 3(a)(1)(A)(i).
    \end{itemize}
    \item[b.] \textbf{Other rights.}
    \begin{itemize}
        \item[1.] Moral rights, such as the right of integrity, are not licensed under this Public License, nor are publicity,
          privacy, and/or other similar personality rights; however, to the extent possible, the Licensor waives and/or agrees not to assert any such rights held by the Licensor to the limited extent necessary to allow You to exercise the Licensed Rights, but not otherwise.
        \item[2.] Patent and trademark rights are not licensed under this Public License.
        \item[3.] To the extent possible, the Licensor waives any right to collect royalties from You for the exercise of the Licensed Rights, whether directly or through a collecting society under any voluntary or waivable statutory or compulsory licensing scheme. In all other cases the Licensor expressly reserves any right to collect such royalties.
    \end{itemize}
\end{itemize}

\begin{center}
    \textbf{Section 3 -- License Conditions.}
\end{center}

Your exercise of the Licensed Rights is expressly made subject to the following conditions.

\begin{itemize}
    \item[a.] \textbf{Attribution.}
    \begin{itemize}
        \item[1.] If You Share the Licensed Material (including in modified form), You must:
        \begin{itemize}
            \item[A.] retain the following if it is supplied by the Licensor with the Licensed Material:
            \begin{itemize}
                \item[i.] identification of the creator(s) of the Licensed Material and any others designated to receive attribution, in any reasonable manner requested by the Licensor (including by pseudonym if designated);
                \item[ii.] a copyright notice;
                \item[iii.] a notice that refers to this Public License;
                \item[iv.] a notice that refers to the disclaimer of warranties;
                \item[v.] a URI or hyperlink to the Licensed Material to the extent reasonably practicable;
            \end{itemize}
            \item[B.] indicate if You modified the Licensed Material and retain an indication of any previous modifications; and
            \item[C.] indicate the Licensed Material is licensed under this Public License, and include the text of, or the URI or hyperlink to, this Public License.
        \end{itemize}
        \item[2.] You may satisfy the conditions in Section 3(a)(1) in any reasonable manner based on the medium, means, and context in which You Share the Licensed Material. For example, it may be reasonable to satisfy the conditions by providing a URI or hyperlink to a resource that includes the required information.
        \item[3.] If requested by the Licensor, You must remove any of the information required by Section 3(a)(1)(A) to the extent reasonably practicable.
    \end{itemize}
    \item[b.] \textbf{ShareAlike.}

     In addition to the conditions in Section 3(a), if You Share Adapted Material You produce, the following conditions also apply.

    \begin{itemize}
        \item[1.] The Adapter's License You apply must be a Creative Commons license with the same License Elements, this version or later, or a BY-SA Compatible License.
        \item[2.] You must include the text of, or the URI or hyperlink to, the Adapter's License You apply. You may satisfy this condition in any reasonable manner based on the medium, means, and context in which You Share Adapted Material.
        \item[3.] You may not offer or impose any additional or different terms or conditions on, or apply any Effective Technological Measures to, Adapted Material that restrict exercise of the rights granted under the Adapter's License You apply.
    \end{itemize}
\end{itemize}

\begin{center}
    \textbf{Section 4 -- Sui Generis Database Rights.}
\end{center}

Where the Licensed Rights include Sui Generis Database Rights that apply to Your use of the Licensed Material:

\begin{itemize}
    \item[a.] for the avoidance of doubt, Section 2(a)(1) grants You the right to extract, reuse, reproduce, and Share all or a substantial portion of the contents of the database;
    \item[b.] if You include all or a substantial portion of the database contents in a database in which You have Sui Generis Database Rights, then the database in which You have Sui Generis Database Rights (but not its individual contents) is Adapted Material, including for purposes of Section 3(b); and
    \item[c.] You must comply with the conditions in Section 3(a) if You Share all or a substantial portion of the contents of the database.
\end{itemize}

For the avoidance of doubt, this Section 4 supplements and does not replace Your obligations under this Public License where the Licensed Rights include other Copyright and Similar Rights.

\begin{center}
    \textbf{Section 5 -- Disclaimer of Warranties and Limitation of Liability.}
\end{center}

\begin{itemize}
    \item[\textbf{a.}] \textbf{Unless otherwise separately undertaken by the Licensor, to the extent possible, the Licensor offers the Licensed Material as-is and as-available, and makes no representations or warranties of any kind concerning the Licensed Material, whether express, implied, statutory, or other. This includes, without limitation, warranties of title, merchantability, fitness for a particular purpose, non-infringement, absence of latent or other defects, accuracy, or the presence or absence of errors, whether or not known or discoverable. Where disclaimers of warranties are not allowed in full or in part, this disclaimer may not apply to You.}
    \item[\textbf{b.}] \textbf{To the extent possible, in no event will the Licensor be liable to You on any legal theory (including, without limitation, negligence) or otherwise for any direct, special, indirect, incidental, consequential, punitive, exemplary, or other losses, costs, expenses, or damages arising out of this Public License or use of the Licensed Material, even if the Licensor has been advised of the possibility of such losses, costs, expenses, or damages. Where a limitation of liability is not allowed in full or in part, this limitation may not apply to You.}
    \item[c.] The disclaimer of warranties and limitation of liability provided above shall be interpreted in a manner that, to the extent possible, most closely approximates an absolute disclaimer and waiver of all liability.
\end{itemize}

\begin{center}
    \textbf{Section 6 -- Term and Termination.}
\end{center}

\begin{itemize}
    \item[a.] This Public License applies for the term of the Copyright and Similar Rights licensed here. However, if You fail to comply with this Public License, then Your rights under this Public License terminate automatically.
    \item[b.] Where Your right to use the Licensed Material has terminated under Section 6(a), it reinstates:
    \begin{itemize}
        \item[1.] automatically as of the date the violation is cured, provided it is cured within 30 days of Your discovery of the violation; or
        \item[2.] upon express reinstatement by the Licensor.
    \end{itemize}

     For the avoidance of doubt, this Section 6(b) does not affect any right the Licensor may have to seek remedies for Your violations of this Public License.

    \item[c.] For the avoidance of doubt, the Licensor may also offer the Licensed Material under separate terms or conditions or stop distributing the Licensed Material at any time; however, doing so will not terminate this Public License.
    \item[d.] Sections 1, 5, 6, 7, and 8 survive termination of this Public License.
\end{itemize}

\begin{center}
    \textbf{Section 7 -- Other Terms and Conditions.}
\end{center}

\begin{itemize}
    \item[a.] The Licensor shall not be bound by any additional or different terms or conditions communicated by You unless expressly agreed.
    \item[b.] Any arrangements, understandings, or agreements regarding the Licensed Material not stated herein are separate from and independent of the terms and conditions of this Public License.
\end{itemize}

\begin{center}
    \textbf{Section 8 -- Interpretation.}
\end{center}

\begin{itemize}
    \item[a.] For the avoidance of doubt, this Public License does not, and shall not be interpreted to, reduce, limit, restrict, or impose conditions on any use of the Licensed Material that could lawfully be made without permission under this Public License.
    \item[b.] To the extent possible, if any provision of this Public License is deemed unenforceable, it shall be automatically reformed to the minimum extent necessary to make it enforceable. If the provision cannot be reformed, it shall be severed from this Public License without affecting the enforceability of the remaining terms and conditions.
    \item[c.] No term or condition of this Public License will be waived and no failure to comply consented to unless expressly agreed to by the Licensor.
    \item[d.] Nothing in this Public License constitutes or may be interpreted as a limitation upon, or waiver of, any privileges and immunities that apply to the Licensor or You, including from the legal processes of any jurisdiction or authority.
\end{itemize}

\end{document}
